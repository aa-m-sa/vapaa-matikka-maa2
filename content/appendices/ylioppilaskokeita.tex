\section{Tehtäviä ylioppilaskokeista}

\subsubsection*{Lyhyen oppimäärän tehtäviä}

k = lyhyt matematiikka
K = pitkä matematiikka

% \begin{tehtava}
% (K2011/1a) \\ Ratkaise yhtälö $4x+(5x-4) = 12+3x$.
% \begin{vastaus}
% $x=\frac{16}{3} = 5\frac13$
% \end{vastaus}
% \end{tehtava}

\begin{tehtava}
(k2011/1b) Sievennä lauseke $x^2+x-(x^2-x)$.
\begin{vastaus}
$2x$
\end{vastaus}
\end{tehtava}

\begin{tehtava}
(k2011/2b) \\ Sievennä lauseke $(\sqrt{x}-1)^2+2\sqrt{x}$.
\begin{vastaus}
$x+1$
\end{vastaus}
\end{tehtava}

\begin{tehtava}
(s2011/1c) \\ Ratkaise yhtälö $x^2-3(x+3) = 3x-18$.
\begin{vastaus}
$x=3$
\end{vastaus}
\end{tehtava}

\begin{tehtava}
(k2012/1a) Ratkaise yhtälö $7x+3 = 31$.
\begin{vastaus}
$x = 4$
\end{vastaus}
\end{tehtava}

\begin{tehtava}
(s2012/1a) Ratkaise yhtälö $x^2-2x = 0$.
\begin{vastaus}
$x=0$ tai $x=2$
\end{vastaus}
\end{tehtava}

\subsubsection*{Pitkän oppimäärän tehtäviä}

\begin{tehtava}
(K2011/1b) Ratkaise epäyhtälö $x^2-2 \leq x$.
\begin{vastaus}
$-1 \leq x \leq 2$
\end{vastaus}
\end{tehtava}

\begin{tehtava}
  (S2011/3b) Ratkaise epäyhtälö $\frac{2x+1}{x-1} \geq 3$.
\begin{vastaus}
$1<x \leq 4$
\end{vastaus}
\end{tehtava}

\begin{tehtava}
(S2012/1a) \\ Ratkaise yhtälö $2(1-3x+3x^2) = 3(1+2x+2x^2)$.
\begin{vastaus}
$x=-\frac{1}{12}$
\end{vastaus}
\end{tehtava}

\begin{tehtava}
(S2013/1a) \\ Ratkaise yhtälö $x^2+6x=2x^2+9$.
\begin{vastaus}
$x=3$
\end{vastaus}
\end{tehtava}

\begin{tehtava}
(S2013/1b) \\ Ratkaise yhtälö $\frac{1+x}{1-x}=\frac{1-x^2}{1+x^2}$.
\begin{vastaus}
$x=0$ tai $x=-1$
\end{vastaus}
\end{tehtava}

\begin{tehtava}
(S2013/1c) \\ Esitä polynomi $x^2-9x+14$ ensimmäisen asteen polynomien tulona.
\begin{vastaus}
$(x-2)(x-7)$
\end{vastaus}
\end{tehtava}
  
% (S2011, 1b) Suorakulmaisen kolmion hypotenuusan pituus on 5 % Kurssi 3
%   ja toisen kateetin pituus 2. Laske toisen kateetin pituus.

  

%fixme: etsi lisää yo-tehtäviä, näitä on kyllä olemassa