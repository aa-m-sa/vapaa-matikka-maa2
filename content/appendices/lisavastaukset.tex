\begin{Vastaus}{180}
        \begin{alakohdat}
            \alakohta{$x^2+4x+4$}
            \alakohta{$x^2-10x+25$}
            \alakohta{$b^2-16$}
            \alakohta{$x^2-2x+1$}
            \alakohta{$4x^2+4x+1$}
            \alakohta{$1-a^2$}
            \alakohta{$x^2-14x+49$}
            \alakohta{$y^2+2y+1$}
            \alakohta{$x^2+6xy+9y^2$}
            \alakohta{$9x^2-6x+1$}
            \alakohta{$4x^2-1$}
            \alakohta{$t^2-4t+4$}
            \alakohta{$4x^2+12x+9$}
            \alakohta{$a^2-4a+4$}
            \alakohta{$25x^2-4$}
            \alakohta{$9-c^2$}
            \alakohta{$100x^2-20x+1$}
            \alakohta{$4a^2+4ab+b^2$}
            \alakohta{$x^2+12x+36$}
        \end{alakohdat}
    
\end{Vastaus}
\begin{Vastaus}{181}
        \begin{alakohdat}
            \alakohta{$(1-a)(1+a)$}
            \alakohta{$(a-2)^2$}
            \alakohta{$(2x+1)^2$}
            \alakohta{$(x-7)^2$}
            \alakohta{$(10x-1)^2$}
            \alakohta{$(2a+b)^2$}
            \alakohta{$(y+1)^2$}
   		     \alakohta{$(x+2)^2$}
            \alakohta{$(2x+1)(2x-1)$}
             \alakohta{$(t-2)^2$}
            \alakohta{$(5x+2)(5x-2)$}
    	    \alakohta{$(b+4)(b-4)$}
 	       \alakohta{$(x-1)^2$}
            \alakohta{$(3-c)(3+c)$}
            \alakohta{$(x+6)^2$}
            \alakohta{$(2x+3)^2$}
  		    \alakohta{$(x-5)^2$}
            \alakohta{$(x-3y)^2$}
            \alakohta{$(3x-1)^2$}
         \end{alakohdat}
    
\end{Vastaus}
\begin{Vastaus}{182}
        \begin{alakohdat}
            \alakohta{$x^4-2x^2+1$}
            \alakohta{$a^{12}+6a^6b^3+9b^6$}
            \alakohta{$-(144-9x^2)=9x^2-144$}
            \alakohta{$x^2+2+\frac{1}{x^2}$}
         \end{alakohdat}
    
\end{Vastaus}
\begin{Vastaus}{183}
    \begin{alakohdat}
            \alakohta{Tutki lukujen $2+\sqrt{3}$ ja $2-\sqrt{3}$ tuloa.}
	        \alakohta{Yleisesti $\left(a+\sqrt{a^2-1}\right)^{-1}= a-\sqrt{a^2-1}$}
    \end{alakohdat}
    
\end{Vastaus}
\begin{Vastaus}{184}
     Aloita tiedosta $\left(x-\frac{1}{x}\right)^2 \geq 0$ ja sievennä.
    
\end{Vastaus}
\begin{Vastaus}{185}
     $x^4\geq 0$ ja $x^2 \geq 0$, joten $f(x) \geq 1$.
    
\end{Vastaus}
\begin{Vastaus}{186}
     Opastus: Aloita tiedosta $\left(\sqrt{a}-\sqrt{b}\right)^2 \geq 0$ ja sievennä. Yhtäsuuruus pätee, kun $a = b$.
    
\end{Vastaus}
\begin{Vastaus}{187}
     $7,7$~cm
    
\end{Vastaus}
\begin{Vastaus}{188}
    $x =\frac{3 \pm \sqrt{13}}{2}$
    
\end{Vastaus}
\begin{Vastaus}{189}
		$10^2+11^2+12^2 = 13^2 + 14^2$.
    	Jos negatiivisetkin luvut sallittaisiin, $(-2)^2+(-1)^2+0^2 = 1^2 + 2^2$ kävisi 			myös. Löytyykö vastaava $4 + 3$ luvun sarja? Entä pidempi?
    
\end{Vastaus}
\begin{Vastaus}{190}
		\begin{alakohdat}
		\alakohta{$1,00$ \euro \ $<$ myyntihinta $<$ $4,5$ \euro.}
		\alakohta{$2,75$ \euro, jolloin voitto on $612,50$ \euro. }
		% Epäilyttävän hyvä bisnes
		\end{alakohdat}
    
\end{Vastaus}
\begin{Vastaus}{191}
	Luvut ovat $-1, 0$ ja $1$ tai $14, 15$ ja $16$.
    
\end{Vastaus}
\begin{Vastaus}{192}
		\begin{alakohdat}
		\alakohta{$-5<x<0$ tai $3 < x$}
		\alakohta{$x<0$ tai $x = 1$}
		\end{alakohdat}
    
\end{Vastaus}
\begin{Vastaus}{193}
	$P(x)>0$ kun $x > \sqrt{a}$ tai $-\sqrt{a}<x<0$.
    
\end{Vastaus}
\begin{Vastaus}{194}
		\begin{alakohdat}
		\alakohta{$3x$}
		\alakohta{$x^3$}
		\alakohta{$-10x^2$}
		\alakohta{$17x^2$}
		\alakohta{$-3x^5$}
		\end{alakohdat}
    
\end{Vastaus}
\begin{Vastaus}{195}
		\begin{alakohdat}
		\alakohta{$3x^3-x^2+3x$}
		\alakohta{$2x-8$}
		\alakohta{$2x^2+2x+7$}
		\alakohta{$3x^2-21x$}
		\alakohta{$-6x^7+24x^4-2x^2$}
		\end{alakohdat}
    
\end{Vastaus}
\begin{Vastaus}{196}
		\begin{alakohdat}
		\alakohta{$x^2+2x-8$}
		\alakohta{$3a^2-2ab-b^2$}
		\alakohta{$a^2+6a+9$}
		\alakohta{$x^2-2x+1$}
		\alakohta{$16x^2-1$}
		\end{alakohdat}
    
\end{Vastaus}
\begin{Vastaus}{197}
		\begin{alakohdat}
		\alakohta{$x(4x+1)$}
		\alakohta{$5xy(x^2+2y)$}
		\alakohta{$(y+3)(y-3)$}
		\alakohta{$(x-2)^2$}
		\alakohta{$(x^3+10)(x-5)$}
		\end{alakohdat}
    
\end{Vastaus}
\begin{Vastaus}{198}
		$x=3$ tai $x=-2$ tai $x=1$.
    
\end{Vastaus}
\begin{Vastaus}{199}
		Koska $x^6\geq 0$ ja $x^2 \geq 0$. (Parilliset potenssit.)
    
\end{Vastaus}
\begin{Vastaus}{200}
		\begin{alakohdat}
		\alakohta{Opastus: Kerro sulut auki.}
		\alakohta{$(x-y)(x^2+xy+y^2)(x+y)(x^2-xy+y^2)$}
		\end{alakohdat}
    
\end{Vastaus}
\begin{Vastaus}{201}
		\begin{alakohdat}
		\alakohta{$x=0$, $x=-2$ ja $x=1$}
		\alakohta{$x<2$ tai $0<x<1$}
		\alakohta{Vähintään 3. (Kuvaajassa näkyvän alueen
		ulkopuolella voisi olla lisää.)}
		\end{alakohdat}
    
\end{Vastaus}
\begin{Vastaus}{202}
		$8(x-2y)(x+2y)(x^2+y^2+7)$. \\
    Opastus: Älä kerro aluksi sulkuja auki vaan käytä heti muistikaavaa.
    
\end{Vastaus}
\begin{Vastaus}{203}
		\begin{alakohdat}
		\alakohta{$x = \frac{7}{3}$}
		\alakohta{$x=-\frac{5}{6}$}
		\alakohta{$ x= \frac{93}{5}=18\frac{3}{5}$}
		\end{alakohdat}
    
\end{Vastaus}
\begin{Vastaus}{204}
		\begin{alakohdat}
		\alakohta{$2 \leq x \leq 7$}
		\alakohta{$-3 < y \leq 0$}
		\alakohta{$z < 5$}
		\end{alakohdat}
    
\end{Vastaus}
\begin{Vastaus}{205}
		\begin{alakohdat}
		\alakohta{$ x > -2$}
		\alakohta{$x < \frac{5}{2}=2\frac{1}{2}$}
		\alakohta{$x \geq \frac{1}{2}$}
		\end{alakohdat}
    
\end{Vastaus}
\begin{Vastaus}{206}
	7 h tai sitä pidemmissä vuokrissa.
    
\end{Vastaus}
\begin{Vastaus}{207}
        $x \leq \frac{a}{a-2}$, kun $a > 2$ \\
        $x \geq \frac{a}{a-2}$, kun $a < 2$ \\
    $x \in \mathbb{R}$, kun $a = 2$ \\
	
\end{Vastaus}
\begin{Vastaus}{208}
		\begin{alakohdat}
		\alakohta{$x= \pm \sqrt{13}$}
		\alakohta{$x=0$ tai $x=-\frac{2}{5}$}
		\alakohta{$x=-3$ tai $x= \frac{1}{2}$}
		\end{alakohdat}
    
\end{Vastaus}
\begin{Vastaus}{209}
		\begin{alakohdat}
		\alakohta{$ -2 < x < 3 $}
		\alakohta{$x \leq 0$ tai $x \geq 5$}
	\end{alakohdat}
    
\end{Vastaus}
\begin{Vastaus}{210}
	Iät ovat 9 ja 21. Yhtälö on $x(30-x)=189$.
	
\end{Vastaus}
\begin{Vastaus}{211}
		$k = 7 \pm 4 \sqrt{3}$
    
\end{Vastaus}
\begin{Vastaus}{212}
		$c=-40$
    
\end{Vastaus}
\begin{Vastaus}{213}
	$-1 < x < -\frac{4}{7}$ tai $2 < x < 5$. Huomioi, että suorakulmion $B$
    sivut ovat positiiviset vain, kun $-1<x<5$.
    
\end{Vastaus}
\begin{Vastaus}{214}
		$x=0$ tai $x=\sqrt[5]{\frac{5}{2}}$
    
\end{Vastaus}
\begin{Vastaus}{215}
		$x=\pm \sqrt{3}$
    
\end{Vastaus}
\begin{Vastaus}{216}
		\begin{alakohdat}
		\alakohta{esimerkiksi $x^4=-1$ }
		\alakohta{esimerkiksi $x^4(x+1)=0$ }
		\alakohta{mahdotonta}
		\end{alakohdat}
    
\end{Vastaus}
\begin{Vastaus}{217}
		$x=0$ tai $x=\sqrt[5]{\frac{5}{2}}$
    
\end{Vastaus}
\begin{Vastaus}{218}
	Luvut, jotka ovat pienempiä kuin $-1$ ja luvut välillä $]0,1[$.
    
\end{Vastaus}
\begin{Vastaus}{219}
		\begin{alakohdat}
		\alakohta{$x=\pm \sqrt{3}$}
		\alakohta{$x=0$ tai $x\frac{-1 \pm \sqrt{21}}{2}$ }
		\end{alakohdat}
    
\end{Vastaus}
\begin{Vastaus}{220}
		\begin{alakohdat}
		\alakohta{$0<x<5$}
		\alakohta{$0<x\frac{1}{2}$ tai $x<-3$}
		\end{alakohdat}
    
\end{Vastaus}
\begin{Vastaus}{221}
		$\sqrt{12}$~m $\approx 3,46$~m
    
\end{Vastaus}
\begin{Vastaus}{222}
		\begin{alakohdat}
			\alakohta{$0$}
			\alakohta{$(x-y)^2$}
		\end{alakohdat}
    
\end{Vastaus}
\begin{Vastaus}{223}
	$x=\frac{6}{\sqrt[5]{2}-1}$
    
\end{Vastaus}
\begin{Vastaus}{224}
	Sivut ovat $5$ ja $12$.
    
\end{Vastaus}
\begin{Vastaus}{225}
	$x \approx 0,69$, $x \approx 1,86$ tai $x \approx 3,03$.
    
\end{Vastaus}
