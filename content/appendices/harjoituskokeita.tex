\subsection*{Harjoituskoe 1}

\begin{enumerate}
\item Ratkaise epäyhtälöt.\\ a) $1-\dfrac{1-x}{6}<x$\\ b) $(x+1)(x^2-2x-1)\geq0$
\item Ratkaise yhtälöt.\\ a) $4x^2-1=0$\\ b) $x^3=-3x$\\ c) $2y^2=y-8$
\item Millä parametrin $k$:n arvoilla yhtälöllä $kx^2-(k+1)x+1=0$ on kaksi erisuurta reaalijuurta? 
\item Muodosta funktio, joka esittää...

\end{enumerate}

\subsection*{Harjoituskoe 2}

\begin{enumerate}
\item Ratkaise yhtälöt.\\ a) $x^2-5x=0$\\ b) $x^4-1=0$\\ c) $(x-1)(x+4) = x(x-5)$
\item Ratkaise epäyhtälöt.\\ a) $x^2-8\geq0$\\ b) $x^2-8\geq(x-3)^2$\\ c) $x^2-6x+9\leq0$
\item Millä vakion $t$ arvoilla yhtälöllä $tx^2+tx-6=0$ ei ole ratkaisuja?
\item Muodosta funktio, joka esittää...
\end{enumerate}


\subsection*{Harjoituskoe 3}

\begin{enumerate}
\item Jaa polynomi $t^5-t^4+t^3-t^2$tekijöihin. Kuinka monta reaalista nollakohtaa polynomilla on?
\item Olkoon $a > 0$. Millä muuttujan $x$ arvoilla funktion $Q(x)=x3−ax^2$ arvot ovat epänegatiivisia?
\item Millä vakion $r$ arvoilla yhtälöllä $rx^2+rx-1=0$ ei ole reaaliratkaisuja?
\item Muodosta funktio, joka esittää...
\end{enumerate}


\subsection*{Harjoituskoe 4}

\begin{enumerate}


\item Ratkaise epäyhtälöt. \\
a) $2x^3 \geq x$ \\
b) $y^2 \leq 3y -9 $

\item Millä vakion $a$ arvoilla yhtälöllä $x^2+ax+a=0$ on kaksoisjuuri?
\item Muodosta funktio, joka esittää...

\item Johda toisen asteen yhtälön ratkaisukaava lähtien yhtälön normaalimuodosta $ax^2+bx+c=0$, missä $a, b$ ja $c$ ovat reaalisia vakioita ja $a \neq 0$.
\end{enumerate}