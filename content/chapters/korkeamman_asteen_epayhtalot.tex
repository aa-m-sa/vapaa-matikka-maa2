\section{Korkeamman asteen epäyhtälö}
\label{kork_ast}

\qrlinkki{http://opetus.tv/maa/maa2/n-asteinen-polynomiepayhtalo/}{Opetus.tv: \emph{N-asteinen polynomiepäyhtälö} (16:19 ja 12:00)}

Korkeamman asteen epäyhtälö, kuten epäyhtälö
\[
x^3 -6x \leq x^2
\]
voinaan ratkaista siirtämällä kaikki termit epäyhtälön toiselle puolelle ja tutkimalla syntyvän polynomin merkkiä:
\begin{align*}
x^3-6x & \leq x^2 & &\ppalkki -x^2 \\
\underbrace{x^3-x^2-6x}_{P(x)} &\leq 0. &&
\end{align*}
Polynomin $P(x)$ merkin selvittämiseksi ratkaistaan sen nollakohdat:
\begin{align*}
    x^3 - x^2-6x &= 0 & &\ppalkki \text{$x$ yhteiseksi tekijäksi} \\
    x(x^2 -x -6) &= 0 & &\ppalkki \text{tulon nollasääntö} \\
    x = 0 \quad \text{tai} \quad & x^2 -x -6 = 0 & &\ppalkki \text{ratkaisukaava} \\
    x= 0 \quad \text{tai} \quad & x=\frac{-(-1) \pm \sqrt{(-1)^2-4\cdot 1 \cdot (-6)}}{2\cdot 1} && \\
    x = -2 \quad \text{tai} \quad & x = 3. &&
\end{align*}
Polynomin $P$ nollakohdat ovat siis $0$, $-2$ ja $3$. Tästä voidaan jatkaa kahdella eri tavalla.

\textbf{Tapa 1: Tekijöihin jako.}

Jaetaan polynomi tekijöihin nollakohtien avulla:
\[
P(x) = x^3 - x^2-6x = x(x^2-x-6) = x(x+2)(x-3).
\]
Tutkitaan kunkin tulon tekijän merkkiä:
\begin{align*}
    x+2>0 & \quad \text{kun} \quad x > -2\\
    x-3>0 & \quad \text{kun} \quad x > 3\\
    x>0 & \quad \text{kun} \quad x > 0.
\end{align*}

Kootaan tulokset \termi{merkkikaavio}{merkkikaavioon}. Merkitään ensin
lukusuoralle polynomin $P$ nollakohdat. Nämä kolme nollakohtaa jakavat
lukusuoran neljään osaan. Taulukoidaan lukusuoran alle kunkin tekijän
merkki kullakin välillä.
\begin{center}
    \begin{merkkikaavio}{3}
        \merkkikaavioKohta{$-2$}
        \merkkikaavioKohta{$0$}
        \merkkikaavioKohta{$3$}

        \merkkikaavioFunktio{$x+2$}
        \merkkikaavioMerkki{$-$}
        \merkkikaavioMerkki{$+$}
        \merkkikaavioMerkki{$+$}
        \merkkikaavioMerkki{$+$}

        \merkkikaavioUusirivi
        \merkkikaavioFunktio{$x-3$}
        \merkkikaavioMerkki{$-$}
        \merkkikaavioMerkki{$-$}
        \merkkikaavioMerkki{$-$}
        \merkkikaavioMerkki{$+$}

        \merkkikaavioUusirivi
        \merkkikaavioFunktio{$x$}
        \merkkikaavioMerkki{$-$}
        \merkkikaavioMerkki{$-$}
        \merkkikaavioMerkki{$+$}
        \merkkikaavioMerkki{$+$}

        \merkkikaavioUusiriviKaksoisviiva
        \merkkikaavioFunktio{$x(x+2)(x-3)$}
        \merkkikaavioMerkki{$-$}
        \merkkikaavioMerkki{$+$}
        \merkkikaavioMerkki{$-$}
        \merkkikaavioMerkki{$+$}
    \end{merkkikaavio}
\end{center}
Merkkikaavion alin rivi saadaan tulon merkkisäännöstä: kolmen negatiivisen luvun tulo on negatiivinen, yhden positiivisen ja kahden negatiivisen tulo positiivinen ja
niin edelleen.
Kaavion viimeiseltä riviltä voidaan nyt lukea vastaus alkuperäiseen kysymykseen: $x^3-x^2-6 \leq 0$, kun $x\leq -2$ tai $0\leq x \leq 3$.

\textbf{Tapa 2: Testipisteet.}

Polynomit ovat jatkuvia funktioita. (Jatkuvuutta käsitellään tarkemmin vasta kurssilla 7.)
Intuitiivisesti jatkuvuudessa on kyse siitä, että funktion kuvaaja on yhtenäinen viiva.
Jatkuvuudesta seuraa, että polynomi ei voi vaihtaa merkkiä kulkematta nollakohdan kautta.
Polynomin merkin nollakohtien välillä saa selville laskemalla funktion
arvon jossakin väliltä otetussa testipisteessä.

Esimerkissä nollakohdat olivat $-2$, $0$ ja $3$. Valitaan niiden välistä ja
ympäriltä testipisteiksi vaikkapa $x=-3$, $x=-1$, $x=1$ ja $x=4$. Tarkistetaan funktion merkki kussakin pisteessä:

\begin{tabular}{c|c|l|c}
Väli & Testipiste & $f(x)=x^3-x^2-6x$ & Funktion merkki \\
\hline
$x < -2$ & $x = -3$ & $(-3)^3 -(-3)^2 - 6(-3) = -18$ & $-$ \\
$-2 <x < 0$ & $x = -1$ & $(-1)^3 -(-1)^2 - 6(-1) =4$ & $+$ \\
$0 <x < 3$ & $x = 1$ & $1^3 -1^2 - 6\cdot 1 =  -6$ & $-$ \\
$3 <x $ & $x = 4$ & $4^3 -4^2 - 6\cdot 4 = 24$ & $+$
\end{tabular}

Vastaukseksi saadaan sama kuin edellä: $x^3-x^2-6 \leq 0$, kun $x\leq -2$ tai $0\leq x \leq 3$.

\begin{esimerkki}
Ratkaise epäyhtälö $x^2-x^4 \leq 0$.
\begin{esimratk}
Jaetaan ensin tekijöihin ja ratkaistaan nollakohdat.
\begin{align*}
x^2+x^4 &=x^2(1-x^2) && \ppalkki \text{ muistikaava }\\
&= x^2(1-x)(1+x) 
\end{align*}
Nollakohdat ovat $x=0$, $x=1$ ja $x=-1$. Tehdään merkkikaavio:
\begin{center}
    \begin{merkkikaavio}{3}
        \merkkikaavioKohta{$-1$}
        \merkkikaavioKohta{$0$}
        \merkkikaavioKohta{$1$}

        \merkkikaavioFunktio{$x^2$}
        \merkkikaavioMerkki{$+$}
        \merkkikaavioMerkki{$+$}
        \merkkikaavioMerkki{$+$}
        \merkkikaavioMerkki{$+$}

        \merkkikaavioUusirivi
        \merkkikaavioFunktio{$1-x$}
        \merkkikaavioMerkki{$+$}
        \merkkikaavioMerkki{$+$}
        \merkkikaavioMerkki{$+$}
        \merkkikaavioMerkki{$-$}

        \merkkikaavioUusirivi
        \merkkikaavioFunktio{$1+x$}
        \merkkikaavioMerkki{$-$}
        \merkkikaavioMerkki{$+$}
        \merkkikaavioMerkki{$+$}
        \merkkikaavioMerkki{$+$}

        \merkkikaavioUusiriviKaksoisviiva
        \merkkikaavioFunktio{$x^2(1-x)(1+x)$}
        \merkkikaavioMerkki{$-$}
        \merkkikaavioMerkki{$+$}
        \merkkikaavioMerkki{$+$}
        \merkkikaavioMerkki{$-$}
    \end{merkkikaavio}
\end{center}
Merkkikaaviosta voidaan lukea, että $x^2(1-x)(1+x) <0$, kun
$x < -1$ tai $x >1$. Lisäksi $x^2(1-x)(1+x)=0$, kun $x=-1$,
$x=0$ tai $x=1$.
\end{esimratk}
\begin{esimvast}
$x \leq -1$, $x=0$ tai $x \geq 1$.
\end{esimvast}
\end{esimerkki}

\begin{tehtavasivu}

\paragraph*{Opi perusteet}

\begin{tehtava}
    Ratkaise
    \begin{alakohdat}
        \alakohta{$(x-1)(x-2)(x-3) \le 0$}
        \alakohta{$(x-1)(x-2)(x-3) > 0$}
        \alakohta{$-3(x-1)(x-2)(x-3) > 0$.}
    \end{alakohdat}
    \begin{vastaus}
        \begin{alakohdat}
            \alakohta{$x \le 1$ tai $2 \le x \le 3$}
            \alakohta{$1 < x < 2$ tai $x>3$}
            \alakohta{$x < 1$ tai $2<x<3$}
        \end{alakohdat}
    \end{vastaus}
\end{tehtava}

\begin{tehtava}
    Ratkaise $x^3-x^2<0$.
    \begin{vastaus}
        $x<0$ tai $0<x<1$
    \end{vastaus}
\end{tehtava}

\begin{tehtava}
    Ratkaise $x^4 \le 1$.
    \begin{vastaus}
        $-1 \le x \le 1$
    \end{vastaus}
\end{tehtava}

\paragraph*{Hallitse kokonaisuus}

%neliö epänegatiivinen
\begin{tehtava}
    Ratkaise $(2x^3+4x^2-5x+7)^2 < 0$.
    \begin{vastaus}
        Ei ratkaisuja.
    \end{vastaus}
\end{tehtava}

%bikvadraattinen
\begin{tehtava}
    Ratkaise $x^4-3x^2-18 \le 0$.
    \begin{vastaus}
        $-\sqrt{6}\le x \le \sqrt{6}$
    \end{vastaus}
\end{tehtava}

\begin{tehtava} % Korkeamman asteen epäyhtälö
Olkoon $a > 0$. Millä muuttujan $x$ arvoilla funktion
$P(x)=x^3-ax$ arvot ovat positiivisia?
    \begin{vastaus}
	$P(x)>0$ kun $x > \sqrt{a}$ tai $-\sqrt{a}<x<0$.
    \end{vastaus}
\end{tehtava}

\begin{tehtava}
    Koska tulolla ja osamäärällä on sama merkkisääntö, merkkikaavioita
	voidaan käyttää myös osamääriin. Ratkaise epäyhtälöt
    \begin{alakohdat}
        \alakohta{$\frac{(x+3)(x-2)}{x-5} \le 0$}
        \alakohta{$x \geq \frac{1}{x}$}
    \end{alakohdat}
    \begin{vastaus}
        \begin{alakohdat}
            \alakohta{$x \le -3$ tai $2 \le x < 5$}
            \alakohta{$-1 \leq x < 0$ tai $x \geq 1$}
    	\end{alakohdat}
    \end{vastaus}
\end{tehtava}


%yhteinen tekijä x^3, binomikaava käänteisesti
\begin{tehtava}
    Ratkaise $4x^5+9 x^3 \le 12 x^4$.
    \begin{vastaus}
        $x\le0$ tai $x=\frac{3}{2}$
    \end{vastaus}
\end{tehtava}

\begin{tehtava}
Ratkaise $(x^5-2)(x^8-1) >0$
\begin{vastaus}
$x > \sqrt[5]{2}$ tai $-1<x<1$
\end{vastaus}
\end{tehtava}

\begin{tehtava}
Epäyhtälöiden ratkaisut/todistukset perustuvat usein tietoon, että epänegatiivisten lukujen summa on epänegatiivinen ja nolla jos, ja vain jos kaikki yhteenlaskettavat ovat nollia.

\begin{alakohdat}
\alakohta{Ratkaise $x^6 + x^2+1 > 0$}
\alakohta{Ratkaise $x^{10} + (x-1)^{10} < 0$}
\alakohta{Todista, että kaikilla reaaliluvuilla $x$ ja $y$
\[
(xy-1)^2+(x^2-y^2)^4+(xy-x-y+1)^6 \geq 0
\]
ja että epäyhtälössä vallitsee yhtäsuuruus jos, ja vain jos $x = y = 1$}
\end{alakohdat}

\begin{vastaus}
\begin{alakohdat}
\alakohta{$x \in \R$}
\alakohta{Epäyhtälöllä ei ole ratkaisuja}
\alakohta{Vinkki: Käytä tehtävänannon havaintoa ja tutki, millä $x$:n ja $y$:n arvoilla summattavat saavat arvon 0}
\end{alakohdat}
\end{vastaus}
\end{tehtava}

\paragraph*{Lisää tehtäviä}

\begin{tehtava} % Korkeamman asteen epäyhtälö
Ratkaise epäyhtälöt
		\begin{alakohdat}
		\alakohta{$x^3 + 2x^2-15x  > 0$  }
		\alakohta{$x^3-2x^2+x \leq 0$  }
		\end{alakohdat}
    \begin{vastaus}
		\begin{alakohdat}
		\alakohta{$-5<x<0$ tai $3 < x$}
		\alakohta{$x<0$ tai $x = 1$}
		\end{alakohdat}
    \end{vastaus}
\end{tehtava}





% x yhteinen tekijä ja sij. y=x^5
\begin{tehtava}
    Ratkaise $x+2x^6+x^{11}<0$.
    \begin{vastaus}
        $x<-1$ tai $ -1<x<0$
    \end{vastaus}
\end{tehtava}

\end{tehtavasivu}
