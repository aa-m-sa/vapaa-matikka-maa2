\section{Diskriminantti}

\qrlinkki{http://opetus.tv/maa/maa2/diskriminantti/}{Opetus.tv: \emph{diskriminantti 2. asteen yhtälölle} (7:56 ja 8:30)}

\begin{esimerkki}
    Ratkaistaan toisen asteen yhtälö $3x^2-5x+10=0$.\\
    
%    \begin{align*}
%        \underbrace{3}_{=a}x^2\underbrace{-5}_{=b}x+\underbrace{10}_{=c}=0
%    \end{align*}
    Sijoitetaan $a=3$, $b=-5$ ja $c=10$ toisen asteen yhtälön ratkaisukaavaan $x=\frac{-b \pm \sqrt[]{b^2-4ac}}{2a}$.
    \begin{align*}
        x &=\frac{-(-5) \pm \sqrt[]{(-5)^2-4\cdot 3 \cdot 10}}{2 \cdot 3} \\
        x &=-\frac{5 \pm \sqrt[]{25-120}}{6} \\
          x &=-\frac{5 \pm \sqrt[]{-95}}{6} 
    \end{align*}
    Koska juurrettava on negatiivinen,
	 yhtälöllä ei ole ratkaisuja.
\end{esimerkki}

Edellisen esimerkin tulokseen päästäisiin nopeamminkin.
Koska ratkaisukaavassa esiintyvä lauseke $b^2-4ac$ on negatiivinen,
ratkaisua ei ole.

Lauseketta $b^2-4ac$ kutsutaan \termi{diskriminantti}{diskriminantiksi}. Sen arvo kertoo yhtälön ratkaisujen lukumäärän. Jos $D<0$, ratkaisuja ei ole, sillä negatiivisella luvulla ei ole neliöjuurta. Jos $D=0$, neliöjuuren arvoksikin tulee $0$ ja ratkaisuja saadaan vain yksi ($\pm 0 = 0$).
Jos $D>0$, neliöjuuri saa positiivisen arvon ja ratkaisuja on kaksi.

Diskriminantin avulla voidaan siis tutkia yhtälön ratkaisujen lukumäärää ilman että
yhtälöä tarvitsee ratkaista.

\newpage
\laatikko{Toisen asteen yhtälön $ax^2+bx+c=0$ ratkaisujen lukumäärä
voidaan laskea diskriminantin $D=b^2-4ac$ avulla seuraavasti:
\begin{itemize}
\item
Jos $D<0$, yhtälöllä ei ole reaalisia ratkaisuja.
\item
Jos $D=0$, yhtälöllä on tasan yksi reaalinen ratkaisu.
\item
Jos $D>0$, yhtälöllä on kaksi erisuurta reaaliratkaisua.
\end{itemize}
}
Tapauksessa $D=0$ yhtälön ainoaa ratkaisua kutsutaan
sen \termi{kaksoisjuuri}{kaksoisjuureksi}.

\begin{esimerkki}
\ \\
\parbox{4.5cm}{
\begin{kuvaajapohja}{1}{-1}{3}{-1}{3}
  \kuvaaja{2*x**2-2*x+1}{}{blue}
\end{kuvaajapohja}
}
\parbox{6cm}{$2x^2-2x+1=0$:\\$D=(-2)^2-4 \cdot 2 \cdot 1=4-8=-4$, eli $D <0$. Ei reaalisia ratkaisuja.}
\\
\parbox{4.5cm}{
\begin{kuvaajapohja}{1}{-1}{3}{-1}{3}
  \kuvaaja{x**2-2*x+1}{}{blue}
\end{kuvaajapohja}
}
\parbox{6cm}{$x^2-2x+1=0$:\\$D=(-2)^2-4 \cdot 1 \cdot 1=4-4=0$, eli $D = 0$. Yksi reaaliratkaisu.}
\\
\parbox{4.5cm}{
\begin{kuvaajapohja}{1}{-1}{3}{-2}{2}
  \kuvaaja{2*x**2-4*x+1}{}{blue}
\end{kuvaajapohja}
}
\parbox{6cm}{$2x^2-4x+1=0$:\\$D=(-4)^2-4 \cdot 2 \cdot 1=16-8=8$, eli $D > 0$. Kaksi eri reaaliratkaisua.}
\end{esimerkki}

\newpage

\begin{esimerkki}
Selvitetään, onko yhtälöllä $x^2+x+2=0$ ratkaisuja.

Tutkitaan diskriminanttia.
\[D=1^2-4\cdot 1 \cdot 2 = 1-8 = -7\]
Koska $D<0$, yhtälöllä ei ole ratkaisuja.

Jos yhtälön ratkaisemista yrittäisi ratkaisukaavan avulla, tulisi
neliöjuuren alle negatiivinen luku.
\end{esimerkki}

\begin{esimerkki}
Millä $k$:n arvolla yhtälöllä $9x^2+kx+1$ on tasan yksi ratkaisu?

Jotta ratkaisuja olisi tasan yksi, on diskriminantin oltava 0.
\begin{align*}
D &= 0\\
k^2-4\cdot 9\cdot 1 &= 0\\
k^2-36 &= 0\\
k^2 &= 36\\
k &= \pm 6
\end{align*}
Yhtälöllä on täsmälleen yksi ratkaisu, kun $k=-6$ tai $k=6$.
\end{esimerkki}

\begin{tehtavasivu}

\paragraph*{Opi perusteet}

\begin{tehtava}
	Laske diskriminanttien arvot.
	Kuinka monta ratkaisua yhtälöillä on?
	\begin{alakohdat}
		\alakohta{$3x^2+4x+1=0$}
		\alakohta{$x^2+2x+5=0$}
		\alakohta{$3x^2-6x+3=0$}
		\alakohta{$x^2-7x-40=0$}
	\end{alakohdat}
	\begin{vastaus}
		\begin{alakohdat}
			\alakohta{$D=4$, eli kaksi ratkaisua.}
			\alakohta{$D=-1$, eli ei ratkaisuja.}
			\alakohta{$D=0$, eli $1$ ratkaisu}
			\alakohta{$D=209$, kaksi ratkaisua}
		\end{alakohdat}
	\end{vastaus}
\end{tehtava}


\begin{tehtava}
	Kuinka monta ratkaisua yhtälöillä on?
	\begin{alakohdat}
		\alakohta{$12x^2+12x-4=0$}
		\alakohta{$12x^2+12x+4=0$}
	\end{alakohdat}
	\begin{vastaus}
		\begin{alakohdat}
			\alakohta{Kaksi. $D=12^2-4 \cdot 12 \cdot (-4) = 336 >0$}
			\alakohta{Ei yhtään. $D=12^2-4 \cdot 12 \cdot 4 = -48 <0$}
		\end{alakohdat}
	\end{vastaus}
\end{tehtava}

\begin{tehtava}
	Tulkitse polynomifunktion lauseketta: Onko kyseessä ylös- vai alaspäin aukeava paraabeli?
	Kuinka monta nollakohtaa funktiolla on?
	\begin{alakohdat}
		\alakohta{$P(x)=-3x^2+9x-5$}
		\alakohta{$Q(y)=5y^2-2y+1$}
		\alakohta{$R(z)=z^2-7z-40$}
		\alakohta{$S(w)=3w^2-6w+3$}
	\end{alakohdat}
	\begin{vastaus}
	Nollakohtien määrä voidaan päätellä diskriminantin arvosta.
		\begin{alakohdat}
			\alakohta{alaspäin, 2 nollakohtaa}
			\alakohta{ylöspäin, ei yhtään nollakohtaa}
			\alakohta{ylöspäin,2 nollakohtaa}
			\alakohta{ylöspäin, 1 nollakohta}
		\end{alakohdat}

	\end{vastaus}


\end{tehtava}


\begin{tehtava}
	Millä luvuilla $c$ yhtälöllä $x^2+5x+c = 0$ ei ole ratkaisua?
	\begin{vastaus}
		 $c> \frac{25}{4} =6,25$. (Halutaan $D < 0$, eli $5^2-4\cdot 1 \cdot c <0$.)
	\end{vastaus}
\end{tehtava}

\paragraph*{Hallitse kokonaisuus}

\begin{tehtava}
	Kuinka monta ratkaisua yhtälöillä on?
	\begin{alakohdat}
		\alakohta{$10x^2-8x-35=14x-10$}
		\alakohta{$-6x^2+15x-59=5x+17$}
%		\alakohta{$7x^2-6x+2=10$}
		\alakohta{$3x^2+7x=2x^2+x-9$}
	\end{alakohdat}
	\begin{vastaus}
		\begin{alakohdat}
			\alakohta{Kaksi. $D=(-22)^2-4 \cdot 10 \cdot (-25) = 1484 >0$}
			\alakohta{Nolla. $D=10^2-4\cdot (-6) \cdot (-76) = -1724 <0$}
%			\alakohta{Kaksi. $D=(-6)^2-4\cdot 7\cdot (-8) 260 > 0$}
			\alakohta{Yksi. $D=6^2-4\cdot 1 \cdot 9 = 0$}
		\end{alakohdat}
	\end{vastaus}
\end{tehtava}

\begin{tehtava}
	Millä vakion $a$ arvoilla yhtälöllä \\ $(2a-1)x^2+(a+1)x+3=0$ on 
	täsmälleen yksi juuri?
	\begin{vastaus}
		Sopivat $a$:n arvot ovat $\frac{1}{2}$, $11+6\sqrt{3}$ ja $11-6\sqrt{3}$.
	\end{vastaus}
\end{tehtava}

\begin{tehtava}
	Mitä voit sanoa ratkaisujen lukumäärästä vaillinaisten yhtälöiden $ax^2+c=0$  ja $ax^2+bx=0$  tapauksessa? Oletetaan, että $a$,$b$,$c \neq 0$.
	\begin{vastaus}
		\begin{description}
			\item[$ax^2+c=0$] Joko kaksi ratkaisua tai ei yhtään ratkaisua. ($D \neq 0$)
			\item[$ax^2+bx=0$] Aina kaksi ratkaisua. ($D > 0$)
		\end{description}
	\end{vastaus}
\end{tehtava}

\begin{tehtava}
	$ \star $ Osoita, että diskriminantti on $0$ jos ja vain jos yhtälö voidaan esittää muodossa \\ $(c_1 x+ c_2)^2=0$, missä $c_1$ ja $c_2$ ovat reaalilukuja.
	\begin{vastaus}
		Suunta "$\Rightarrow$": $(c_1 x+ c_2)^2=0 \Leftrightarrow c_1^2 x^2 + 2c_1 c_2 x+ c_2^2 =0 \Rightarrow
		D=(2 c_1 c_2)^2-4 c_1^2 c_2^2 =4 c_1^2 c_2^2 -4 c_1^2 c_2^2 =0$ \\
		Suunta "$\Leftarrow$": $D=0 \Leftrightarrow b^2-4ac=0 \Leftrightarrow b^2=4ac \Leftrightarrow c=\frac{b^2}{4a} \Rightarrow ax^2+bx+\frac{b^2}{4a}=0 \Leftrightarrow 4a^2x^2+4abx+b^2=0 \Leftrightarrow (2ax+b)^2=0$
	\end{vastaus}
\end{tehtava}

\paragraph*{Lisää tehtäviä}

\begin{tehtava}
	Kuinka monta ratkaisua yhtälöillä on?
	\begin{alakohdat}
		\alakohta{$9x^2+12x-4=0$}
		\alakohta{$5x^2+4x-10=0$}
		\alakohta{$3x^2-12x+12=0$}
		\alakohta{$5x^2+10x-30=0$}
	\end{alakohdat}
	\begin{vastaus}
		\begin{alakohdat}
			\alakohta{Kaksi. $D=12^2-4 \cdot 9 \cdot (-4) = 288 >0$}
			\alakohta{Kaksi. $D=4^2-4\cdot 5 \cdot (-10) = 216 >0$}
			\alakohta{Yksi. $D=(-12)^2-4\cdot 3\cdot 12 =0$}
			\alakohta{Kaksi. $D=10^2-4\cdot 5 \cdot (-30) = 700 >0$}
		\end{alakohdat}
	\end{vastaus}
\end{tehtava}

\begin{tehtava}
	Millä vakion $k$ arvoilla yhtälöllä \\ $-x^2-x-k = 0$ on ratkaisuja?
	\begin{vastaus}
		Pitää olla $D=(-1)^2-4 \cdot (-1) \cdot (-k) \geq 0$. Siis $k \leq \frac{1}{4}$.
	\end{vastaus}
\end{tehtava}

\begin{tehtava}
	Millä vakion $a$ arvoilla yhtälöllä \\ $ax^2+x=ax-5$ on 
	täsmälleen yksi ratkaisu?
	\begin{vastaus}
		$a =11 \pm 2\sqrt{30}$.
	\end{vastaus}
\end{tehtava}

\begin{tehtava}
	Kuinka monta ratkaisua yhtälöillä on vakion $a$ eri arvoilla?
	\begin{alakohdat}
		\alakohta{$x^2+6x+a+1=0$}
		\alakohta{$ax^2+4x-1=0$}
	\end{alakohdat}
	\begin{vastaus}
		\begin{alakohdat}
			\alakohta{Kaksi, kun $a<8$, yksi, kun $a=8$, ja nolla, kun $a>8$.}
			\alakohta{Kaksi, kun $-4<a$ ja $a\neq 0$; yksi, kun $a=-4$ tai $a=0$, ja nolla, kun $a<-4$.}
		\end{alakohdat}
	\end{vastaus}
\end{tehtava}

\end{tehtavasivu}
