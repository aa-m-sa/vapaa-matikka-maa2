\section{Tekijöihinjako}


%Pitkän matematiikan 1. kurssilla on käsitelty lukujen jakamista tekijöihin.


Luvun $12$ \termi{tekijä}{tekijät} luonnolisten lukujen joukossa ovat luvut $1$, $2$, $3$, $4$, $6$ ja $12$. Ne ovat
lukuja, joista saadaan luku $12$ kertomalla ne jollain luonnollisella luvulla. Toisin sanottuna luku $12$ voidaan jakaa
millä tahansa näistä luvuista ilman jakojäännöstä.
%Sanotaan myös, että luvun $12$ \termi{alkutekijä}{alkutekijät} ovat $2$ ja $3$, koska luku $12$ voidaan
%ilmaista niiden tulona ($2\cdot 2\cdot 3 = 2^2\cdot 3 = 12$), mutta näitä tekijöitä ei
%voi enää jakaa pienempiin osatekijöihin. Kokonaisluvun tekijät ovat kokonaislukuja ja alkutekijät alkulukuja.

\begin{esimerkki}
Luku $12$ voidaan kirjoittaa tekijöidensä tulona monella eri tavalla. Esimerkiksi
\begin{align*}
&12 = 2 \cdot 6, \\
&12= 4 \cdot 3 \text{ tai } \\
&12= 2 \cdot 2 \cdot 3.
\end{align*}
\end{esimerkki}

\subsection*{Polynomin jakaminen tekijöihin}

\qrlinkki{http://opetus.tv/maa/maa2/polynomin-jakaminen-tekijoihin/}{Opetus.tv: \emph{polynomin jakaminen tekijöihin} (9:50 ja 5:44)}

Polynomeja voidaan jakaa tekijöihin kuten lukujakin. Polynomien tapauksessa tekijöihinjako tarkoittaa
polynomin esittämistä saman- tai pienempiasteisten polynomien tulona. Aste on aina pienempi, ellei kyse ole pelkästä
vakiokertoimen ottamisesta yhteiseksi tekijäksi.

\subsubsection*{Yhteinen tekijä}

Kun polynomin jokaisessa termissä on sama tekijä, se voidaan ottaa yhteiseksi tekijäksi.

\begin{esimerkki}
Jaa tekijöihin polynomi $10x^3-20x^2$.
\begin{align*}
& 10x^3-20x^2 &\emph{otetaan $10$ yhteiseksi tekijäksi}\\
=& 10(x^3-2x^2) &\emph{otetaan $x^2$ yhteiseksi tekijäksi} \\
=& 10x^2(x-2)  
\end{align*}
\end{esimerkki}

\begin{esimerkki}
Jaa tekijöihin \quad a) $x^3+x$ \quad b) $3x^2+6x.$
\begin{alakohdat}
    \alakohta{$x^3+x = x(x^2+1)$}
    \alakohta{$3x^2+6x = 3x(x+2)$.}
\end{alakohdat}
\end{esimerkki}

\subsubsection*{Muistikaavat}

\begin{esimerkki}
Jaa tekijöihin \quad a) $x^2-4$ \quad b) $x^2+8x+16.$
Jaetaan polynomit tekijöihin hyödyntämällä muistikaavoja
\begin{alakohdat}
    \alakohta{$x^2-4 = x^2-2^2 = (x+2)(x-2)$}
    \alakohta{$x^2+8x+16 = x^2+ 2\cdot 4 \cdot x + 4^2 = (x+4)^2$}
\end{alakohdat}
\end{esimerkki}

\begin{esimerkki}
Jaa tekijöihin polynomi $5x^3-20x^2+20x$.
\begin{align*}
& 5x^3-20x^2+20x \ \ \ \ &\emph{otetaan $5$ yhteiseksi tekijäksi}\\
=& 5(x^3-4x^2+4x)&\emph{otetaan $x$ yhteiseksi tekijäksi}  \\
=& 5x(x^2-4x+4) &\emph{sovelletaan muistikaavaa} \\
=& 5x(x^2-2\cdot 2x+2^2)  \\
=& 5x(x-2)^2
\end{align*}
\end{esimerkki}



%On suositeltavaa tarkistaa itse, että yllä esitetyt tekijöihinjaot todella toimivat. Polynomien tekijöihinjaon toimivuus
%on helppoa tarkistaa -- täytyy vain laskea väitettyjen tekijöiden tulo ja katsoa, onko se alkuperäinen polynomi. Vaikka
%tarkistus onkin helppoa, tässä vaiheessa ei luultavasti vielä ole selvää, miten tekijöihinjaon voisi saada selville
%-- paitsi toisinaan arvaamalla, mutta tähän kysymykseen vastataan myöhemmin tällä kurssilla.

%Polynomien tekijöihinjako ei ole yksiselitteinen, mutta monesti hyödyllisintä on jakaa polynomi tekijöihin samoin kuin
%esimerkkitapauksissa eli niin, että ensimmäisenä on vakiotermi ja kaikissa muissa tekijäpolynomeissa korkeimman asteen
%termin kerroin on 1.
%
%Esimerkki selkeyttänee asiaa. Polynomi $6x^2+30x+36$ voidaan jakaa tekijöihin vaikkapa seuraavilla tavoilla:
%
%\begin{esimerkki}
%\qquad \\
%\begin{itemize}
%    \item $6(x+2)(x+3)$
%    \item $3(2x+4)(x+3)$
%    \item $3(x+2)(2x+6)$
%    \item $2(3x+6)(x+2)$
%    \item $(6x+12)(x+3)$
%    \item $(\frac12 x+1)(12x+36)$
%\end{itemize}
%\end{esimerkki}
%
%Kaikki nämä tavat ovat ''oikein,'' mutta lähes aina ensimmäinen muoto $6(x+2)(x+3)$ on kätevin.
%
%Toisinaan polynomeille voi löytää tekijöitä soveltamalla joitakin seuraavista keinoista:
%
%\begin{itemize}
%\item Otetaan korkeimman asteen termin kerroin yhteiseksi tekijäksi: \\
%$5x^4+3x^2+x-9 = 5(x^4+\frac{3}{5} x^2+\frac{1}{5} x-\frac{9}{5})$
%\item Otetaan $x$ tai sen potenssi yhteiseksi tekijäksi, jos mahdollista: \\
%$x^5+x^3+3x = x(x^4+x^2+3)$
%$x^7+x^6+5x^4+2x^2 = x^2(x^5+x^4+5x^2+2)$
%\item Sovelletaan muistikaavaa käänteisesti \\
%$x^2-5=x^2-\sqrt{5}^2=(x+\sqrt{5})(x-\sqrt{5})$ \\
%$x^2+8x+16=x^2+2\cdot 4x+4^2=(x+4)^2$ \\
%$x^2+x+\frac14=x^2+2\cdot \frac12 x+(\frac12)^2=(x+\frac12)^2$
%\end{itemize}

%Kaikkien polynomien tekijöihinjako ei kuitenkaan näillä menetelmillä onnistu. Myöhemmin tässä kirjassa opitaan, miten toisen asteen polynomin voidaan jakaa tekijöihin nollakohtiensa avulla.

\subsubsection*{Ryhmittely}

Seuraavassa esimerkissä tekijöihin jako on toteutettu termien ryhmittelyn avulla. Se on joissain tapauksissa näppärä tapa jakaa polynomi tekijöihin, mutta oikean ryhmittelyn keksimiseen ei ole helppoa sääntöä.

\begin{esimerkki}
Jaa tekijöihin $x^3+3x^2+x+3$.
\begin{align*}
x^3+3x^2+x+3 &=x^2(x+3)+1(x+3) \\ &=(x^2+1)(x+3)
\end{align*}
\end{esimerkki}

\begin{esimerkki}
Jaa tekijöihin $x^{11}+2x^{10}+3x+6$.
\begin{align*}
& x^{11}+2x^{10}+3x+ &6=x^{10}(x+2)+3(x+2)\\ &=(x^{10}+3)(x+2)
\end{align*}
\end{esimerkki}

\subsubsection*{Yhtälön ratkaisu tekijöihin jakamalla}

Tulon nollasääntö on yksi tärkeimmistä syistä siihen, miksi polynomien tekijöihinjako on hyödyllistä.

Jos vaikkapa haluamme ratkaista yhtälön $2x^3-14x^2+32x-24=0$ ja satumme tietämään, että $2x^3-14x^2+32x-24=2(x-3)(x-2)^2$,
voimme helposti päätellä, että polynomi saa arvon $0$ jos ja vain jos $x-3=0$ tai $x-2=0$. Yhtälön ainoat ratkaisut ovat siis $x=3$ ja $x=2$.


\begin{esimerkki}
Ratkaise yhtälö $x^3-2x^2=0$ tulon nollasäännön avulla.
\begin{esimratk}
\begin{align*}
x^3-2x^2 &= 0 && \ppalkki \text{ otetaan } x^2 \text{ yhteiseksi tekijäksi} \\
x^2\cdot(x-2) &= 0 && \ppalkki \text{ tulon nollasääntö} \\
x^2=0 \textrm{\quad tai}& \quad x-2=0 \\
x=0 \textrm{\quad tai}& \quad x=2 \\
\end{align*}
\end{esimratk}
\begin{esimvast}
$x=0$ tai $x=2$.
\end{esimvast}
\end{esimerkki}

%\begin{esimerkki}
%Ratkaise yhtälö $6x^3-36x^2+54x=0$ tulon nollasäännön avulla.
%\begin{esimratk}
%\begin{align*}
%6x^3-36x^2+54x &= 0 \\
%6(x^3-6x^2+9x) &= 0 && \ppalkki \text{ otetaan } 6 \text{ yhteiseksi tekijäksi} \\
%6x(x^2-6x+9) &= 0 && \ppalkki \text{ otetaan } x \text{ yhteiseksi tekijäksi} \\
%6x(x^2-2\cdot 3\cdot x+3^2)  &= 0 \\
%6x(x-3)^2 &= 0 & \\
%6x(x-3)(x-3) &= 0 && \ppalkki \text{ tulon nollasääntö}\\
%6x=0 \textrm{\quad tai}& \quad x-3=0 \\
%x=0 \textrm{\quad tai}& \quad x=3 \\
%\end{align*}
%\end{esimratk}
%\begin{esimvast}
%$x=0$ tai $x=3$.
%\end{esimvast}
%\end{esimerkki}

Myöhemmin tällä kurssilla esitellään polynomien jakolause, joka antaa syvällisemmän yhteyden polynomien nollakohtien ja tekijöiden välille.

\subsubsection*{Murtolausekkeiden sieventäminen}

Kun murtolausekkeen osoittaja ja nimittäjä jaetaan tekijöihinsä,
kummassakin esiintyvät tekijät voidaan supistaa pois.

\begin{esimerkki}
    Sievennä \quad 
    a) $\dfrac{4x+2y}{6}$ \quad
    b)$\dfrac{x^2+x}{2x+2}$ \quad
    c) $\dfrac{x^2-16}{x+4}$
    \begin{esimratk}
        \begin{alakohdat}
            \alakohta{$\dfrac{4x+2y}{6}=\dfrac{2(2x+y)}{2\cdot 3}=\dfrac{2x+y}{3}$}
            \alakohta{$\dfrac{x^2+x}{2x+2}=\dfrac{x(x+1)}{2(x+1)}=\dfrac{x}{2}$}
            \alakohta{Käytetään muistikaavaa: $\dfrac{x^2-16}{x+4}=\dfrac{(x+4)(x-4)}{x+4} = x-4$.}
        \end{alakohdat}
    \end{esimratk}
    \begin{esimvast}
        a) $\dfrac{2x+y}{3}$ \quad
        b) $\dfrac{x}{2}$ \quad
        c) $x-4$.
    \end{esimvast}
\end{esimerkki}

\begin{tehtavasivu}

\paragraph*{Opi perusteet}

\begin{tehtava}
    Esitä tulona ottamalla yhteinen tekijä.
    \begin{alakohdat}
        \alakohta{$2x+6$}
        \alakohta{$x^2 -4x$}
        \alakohta{$3x^2 - 6x$}
    \end{alakohdat}
    \begin{vastaus}
        \begin{alakohdat}
        \alakohta{$2(x+3)$}
        \alakohta{$x(x-4)$}
        \alakohta{$3x(x-2)$}
        \end{alakohdat}
    \end{vastaus}
\end{tehtava}

\begin{tehtava}
    Jaa tekijöihin.
    \begin{alakohdat}
        \alakohta{$10a+5ab$}
        \alakohta{$x^4 -x^3$}
        \alakohta{$xy+x^2y$}
    \end{alakohdat}
    \begin{vastaus}
        \begin{alakohdat}
        \alakohta{$5a(2+b)$}
        \alakohta{$x^3(x-1)$}
        \alakohta{$xy(1+x)$}
        \end{alakohdat}
    \end{vastaus}
\end{tehtava}

\begin{tehtava}
    Sievennä.
    \begin{alakohdat}
        \alakohta{$\dfrac{3x-9}{3}$}
        \alakohta{$\dfrac{x^2-4x}{5x}$}
        \alakohta{$\dfrac{ab+a}{b^2+b}$}
    \end{alakohdat}
    \begin{vastaus}
        \begin{alakohdat}
        \alakohta{$x-3$}
        \alakohta{$\frac{x-4}{3}$}
        \alakohta{$\frac{a}{b}$}
        \end{alakohdat}
    \end{vastaus}
\end{tehtava}

\paragraph*{Hallitse kokonaisuus}

%\begin{tehtava}
%    Jaa tekijöihin.
%    \begin{alakohdat}
%    	\alakohta{$x^3 - x$}
%        \alakohta{$x^2 - x + \frac{1}{4}$}
%        \alakohta{$9-x^4$}
%    \end{alakohdat}
%    \begin{vastaus}
%        \begin{alakohdat}
%            \alakohta{$x(x-1)^2$}
%            \alakohta{$(x-\frac{1}{2})^2$}
%            \alakohta{$(3+x^2)(3-x^2)$}
%        \end{alakohdat}
%    \end{vastaus}
%\end{tehtava}


\begin{tehtava}
    Jaa tekijöihin.
    \begin{alakohdat}
        \alakohta{$-15x^5 +10y$}
        \alakohta{$x^3y^2 +x^2y^3$}
        \alakohta{$-4a^3 -2a^2 +2ab$}
    \end{alakohdat}
    \begin{vastaus}
        \begin{alakohdat}
        \alakohta{joko $5(-3x^5 +2y)$ tai $-5(3x^5 -2y)$}
        \alakohta{$x^2y^2(x+y)$}
        \alakohta{joko $2a(-2a^2 -a +b)$ tai $-2a(2a^2 +a -b)$}
        \end{alakohdat}
    \end{vastaus}
\end{tehtava}

\begin{tehtava}
    Jaa tekijöihin muistikaavojen avulla.
    \begin{alakohdat}
        \alakohta{$x^2+6x+9$}
        \alakohta{$y^2 - 2y+1$}
        \alakohta{$x^2 -25$}
    \end{alakohdat}
    \begin{vastaus}
        \begin{alakohdat}
        \alakohta{$(x+3)^2$}
        \alakohta{$(y-1)^2$}
        \alakohta{$(x-5)(x+5)$}
        \end{alakohdat}
    \end{vastaus}
\end{tehtava}

\begin{tehtava}
    Jaa tekijöihin ryhmittelemällä sopivasti.
    \begin{alakohdat}
        \alakohta{$x^3 +x^2 +x +1$}
        \alakohta{$a^3 +a^2b +2a +2b$}
        \alakohta{$8m^6-2m^4+4m^2-1$}
    \end{alakohdat}
    \begin{vastaus}
        \begin{alakohdat}
        \alakohta{$(x^2+1)(x+1)$}
        \alakohta{$(a^2+2)(a+b)$}
        \alakohta{$(2m^4 +1)(4m^2 -1)=(2m^4 +1)(2m+1)(2m-1)$}
        \end{alakohdat}
    \end{vastaus}
\end{tehtava}

\begin{tehtava}
    Sievennä.
    \begin{alakohdat}
        \alakohta{$\dfrac{x^3-2x^2}{2-x}$}
        \alakohta{$\dfrac{x^2+6x+9}{x^2+3x}$}
        \alakohta{$\dfrac{4-x^2}{x^2-2x}$}
    \end{alakohdat}
    \begin{vastaus}
        \begin{alakohdat}
        \alakohta{$-x^2$}
        \alakohta{$\frac{x+3}{x}$}
        \alakohta{$-\frac{x+2}{x}$}
        \end{alakohdat}
    \end{vastaus}
\end{tehtava}

\begin{tehtava}
    Jaa tekijöihin.
    \begin{alakohdat}
    	\alakohta{$x^2 -4$}
    	\alakohta{$x^2 -3$}
    	\alakohta{$5x^2 -3$}
		\alakohta{$16-x^4$}
    \end{alakohdat}
    \begin{vastaus}
        \begin{alakohdat}
            \alakohta{$(x+2)(x-2)$}
            \alakohta{$(x+\sqrt{3})(x-\sqrt{3})$}
            \alakohta{$(\sqrt{5}x+\sqrt{3})(\sqrt{5}x-\sqrt{3})$}
            \alakohta{$(4+x^2)(4-x^2)=(4+x^2)(2-x)(2+x)$}
        \end{alakohdat}
    \end{vastaus}
\end{tehtava}

\begin{tehtava}
	Jaa tekijöihin.
	\begin{alakohdat}
		\alakohta{$x^3-x$}
		\alakohta{$5ab+ b+10a+2$}
		\alakohta{$16x^2y^2+8xy+1$}
	\end{alakohdat}
	\begin{vastaus}
		\begin{alakohdat}
			\alakohta{$x(x+1)(x-1)$}
			\alakohta{$(5a+1)(b+2)$}
			\alakohta{$(4xy+1)^2$}
		\end{alakohdat}
	\end{vastaus}
\end{tehtava}

\begin{tehtava}
	Ratkaise yhtälö jakamalla tekijöihin.
	\begin{alakohdat}
		\alakohta{$x^2-16 = 0$}
		\alakohta{$x^2+7x = 0$}
		\alakohta{$x^2-6x+9 = 0$}
	\end{alakohdat}
	\begin{vastaus}
		\begin{alakohdat}
			\alakohta{$x=4$ tai $x=-4$}
			\alakohta{$x=0$ tai $x=-7$}
			\alakohta{$x=3$}
		\end{alakohdat}
	\end{vastaus}
\end{tehtava}

\begin{tehtava} 
Jaa tekijöihin \\ $(3x^2-7y^2+5)^2-(x^2-9y^2-5)^2$.
    \begin{vastaus}
		$8(x-2y)(x+2y)(x^2+y^2+7)$. \\
    Opastus: Älä kerro aluksi sulkuja auki vaan käytä heti muistikaavaa.
    \end{vastaus}
\end{tehtava}

\paragraph*{Lisää tehtäviä}

\begin{tehtava}
    Jaa tekijöihin muistikaavojen avulla.
    \begin{alakohdat}
        \alakohta{$x^2-4x+4$}
        \alakohta{$9y^2 + 6y+1$}
        \alakohta{$49-4x^2$}
    \end{alakohdat}
    \begin{vastaus}
        \begin{alakohdat}
        \alakohta{$(x-2)^2$}
        \alakohta{$(3y+1)^2$}
        \alakohta{$(7-2x)(7+2x)$}
        \end{alakohdat}
    \end{vastaus}
\end{tehtava}


\begin{tehtava}
	Ratkaise yhtälö jakamalla tekijöihin.
	\begin{alakohdat}
		\alakohta{$x^3-x^2 = 0$}
		\alakohta{$x^3+3x^2-4x-12 = 0$}
		\alakohta{$x^2-4x+4 = 4$}
	\end{alakohdat}
	\begin{vastaus}
		\begin{alakohdat}
			\alakohta{$x=0$ tai $x=1$}
			\alakohta{$x=-3$, $x=2$ tai $x=-2$}
			\alakohta{$x=0$ tai $x=4$}
		\end{alakohdat}
	\end{vastaus}
\end{tehtava}

\begin{tehtava}
	Ratkaise yhtälöt.
	\begin{alakohdat}
		\alakohta{$-x^4+4x^2=0$}
		\alakohta{$x^5-16x^3=0$}
	\end{alakohdat}
	\begin{vastaus}
		\begin{alakohdat}
			\alakohta{$x=-2$, $x=0$ tai $x=2$ (Tekijöihin jakamalla yhtälö sievenee muotoon $x^2(2+x)(2-x)=0$.)}
			\alakohta{$x=-4$, $x=0$ tai $x=4$ (Tekijöihin jakamalla yhtälö sievenee muotoon $x^3(x+4)(x-4)=0$.)}
		\end{alakohdat}
	\end{vastaus}
\end{tehtava}

\end{tehtavasivu}
