\section{Toisen asteen polynomifunktio}

\qrlinkki{http://opetus.tv/maa/maa2/toisen-asteen-polynomifunktio/}{Opetus.tv: \emph{toisen asteen polynomifunktio} (7:59)}

Toisen asteen polynomifunktio on muotoa
\begin{align*}
P(x)=ax^2+bx+c,
\end{align*}
missä vakiot $b$ ja $c$ voivat olla mitä tahansa reaalilukuja $(b, \ c \in \R)$ ja $a$ voi olla mikä tahansa reaaliluku, paitsi luku nolla $(a \in \R, \ a \neq 0)$.

Toisen asteen polynomifunktion kuvaajaa nimitetään \termi{paraabeli}{paraabeliksi}.
Kaikki paraabelit ovat keskenään yhdenmuotoisia, vain niiden paikka ja asento vaihtelee.

Funkiton $P(x)=ax^2+bx+c$ kuvaaja on joka ylös- tai alaspäin aukeava paraabeli.
Paraabelin suuntaan vaikuttaa ainoastaan toisen asteen termin kerroin $a$: kun
$a>0$, paraabeli aukeaa ylöpäin, ja kun $a<0$, paraabeli aukeaa alaspäin.

\begin{center}
\begin{tabular}{cc}

\begin{tabular}{c}
	\begin{lukusuora}{-2}{2}{4}
	\lukusuoraisobbox
	\lukusuorakuvaaja{x**2-1}
	\end{lukusuora}
	\\ ylöspäin aukeava paraabeli, \\ $a > 0$
\end{tabular}
&
\begin{tabular}{c}
	\begin{lukusuora}{-2}{2}{4}
	\lukusuoraisobbox
	\lukusuorakuvaaja{-x**2+1}
	\end{lukusuora}
	\\ alaspäin aukeava paraabeli, \\ $a < 0$
\end{tabular}

\end{tabular}
\end{center}

Vakiotermi $c$ vaikuttaa kuvaajan korkeuteen.

%FIXME: kuvaajassa x^2+3 ja x^2 menevät päällekäin 
\begin{luoKuva}{paraabelit}
	kuvaaja.pohja(-5, 5, -1, 8, leveys=7)
	
	kuvaaja.piirra("x**2+3", nimi="$x^2+3$")
	kuvaaja.piirra("x**2", nimi="$x^2$")
\end{luoKuva}

\begin{luoKuva}{paraabelit2}
	kuvaaja.pohja(-5, 5, -4, 6, leveys=7)
	
	kuvaaja.piirra("x**2-3*x", nimi="$x^2-3x$")
	kuvaaja.piirra("x**2", nimi="$x^2$")
\end{luoKuva}


%\begin{luoKuva}{vakio}
%	kuvaaja.pohja(-5, 5, -5, 5, leveys=7)
%	
%	kuvaaja.piirra("*x**2+2", nimi="$x^2+2$")
%	kuvaaja.piirra("x**2", nimi="$x^2$")
%\end{luoKuva}

\begin{center}
	\naytaKuva{paraabelit}
\end{center}

Ensimmäisen asteen termin kerroin $b$ vaikuttaa huipun sijaintiin sekä pysty- että vaakasuunnassa.

\begin{center}
	\naytaKuva{paraabelit2}
\end{center}

%\begin{center}
%	\naytaKuva{vakio}
%\end{center}


%
%\begin{luoKuva}{ekaaste}
%	kuvaaja.pohja(-5, 5, -5, 5, leveys=7)
%	
%	kuvaaja.piirra("*x**2-3*x", nimi="$x^2-3x$")
%	kuvaaja.piirra("x**2", nimi="$x^2$")
%\end{luoKuva}
%
%\begin{center}
%	\naytaKuva{ekaaste}
%\end{center}



\begin{tehtavasivu}

\paragraph*{Opi perusteet}

\begin{tehtava}
  Aukeavatko seuraavat paraabelit ylös- vai alaspäin?
  \begin{alakohdat}
    \alakohta{$4x^2 + 100x - 3$}
    \alakohta{$-x^2 + 1337$}
    \alakohta{$5x^2 - 7x + 5$}
    \alakohta{$-6(-3x^2 + 5)$}
    \alakohta{$-13x(9 - 17x)$}
    \alakohta{$100(1-x^2)$}
  \end{alakohdat}

  \begin{vastaus}
    \begin{alakohdat}
      \alakohta{Ylös}
      \alakohta{Alas}
      \alakohta{Ylös}
      \alakohta{Ylös}
      \alakohta{Ylös}
      \alakohta{Alas}
    \end{alakohdat}
  \end{vastaus}
\end{tehtava}

\begin{tehtava}
Funktiot $P(x)$ ja $Q(x)$ ovat toisen asteen polynomeja.\\
\begin{kuvaajapohja}{1}{-2}{3}{-1}{3}
\kuvaaja{x*(2-x)}{$P(x)$}{black}
\kuvaaja{x**2+1}{$Q(x)$}{black}
\end{kuvaajapohja} \\
Päättele kuvaajan perusteella
\begin{alakohdat}
\alakohta{mihin suuntaan paraabelit aukeavat}
\alakohta{funktion $P$ nollakohdat}
\alakohta{yhtälön $Q(x)=2$ ratkaisu}
\alakohta{polynomin $Q(x)$ vakiotermi}
\end{alakohdat}

\begin{vastaus}
\begin{alakohdat}
\alakohta{$P$ alaspäin, $Q$ ylöspäin.}
\alakohta{nollakohdat: $x=0$ ja $x=2$}
\alakohta{$x=-1$ tai $x=1$}
\alakohta{$1$, sillä kun $x=0$, $Q(x)=1$.}
\end{alakohdat}
\end{vastaus}
\end{tehtava}

\paragraph*{Hallitse kokonaisuus}

\begin{tehtava}
Kuvassa on funktion $P(x)=x^2$ kuvaaja.\\
\begin{kuvaajapohja}{1.5}{-1.5}{1.5}{-1}{3}
\kuvaaja{x**2}{$P(x)=x^2$}{black}
\end{kuvaajapohja} \\
Hahmottele kuvaajan avulla funktioiden
\begin{alakohdat}
\alakohta{$x^2-1$}
\alakohta{$2-x^2$}
\alakohta{$\frac{1}{2}x^2$}
\alakohta{$(x-2)^2$}
\end{alakohdat}
kuvaajat.
\begin{vastaus}
\begin{alakohdat}
\alakohta{
\begin{kuvaajapohja}{1}{-1.5}{1.5}{-2}{2}
\kuvaaja{x**2-1}{$P(x)=x^2-1$}{black}
\end{kuvaajapohja}}
\alakohta{
\begin{kuvaajapohja}{1}{-1.5}{1.5}{-1}{3}
\kuvaaja{2-x**2}{$P(x)=2-x^2$}{black}
\end{kuvaajapohja}}
\alakohta{
\begin{kuvaajapohja}{1}{-1.5}{1.5}{-1}{3}
\kuvaaja{0.5*x**2}{$P(x)=\frac12x^2$}{black}
\end{kuvaajapohja}}
\alakohta{
\begin{kuvaajapohja}{1}{-0.5}{3.5}{-1}{3}
\kuvaaja{(x-2)**2}{$P(x)=(x-2)^2$}{black}
\end{kuvaajapohja}}
\end{alakohdat}
\end{vastaus}
\end{tehtava}

\end{tehtavasivu}