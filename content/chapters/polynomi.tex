\section{Käsitteitä}

\qrlinkki{http://opetus.tv/maa/maa2/polynomien-peruskasitteet/}
{Opetus.tv: \emph{polynomien peruskäsitteet} (9:00)}

\qrlinkki{http://opetus.tv/maa/maa2/polynomin-tasmallinen-maaritelma/}
{Opetus.tv: \emph{polynomin täsmällinen määritelmä} (6:10)}

\subsection*{Polynomit}

\termi{polynomi}{Polynomit} ovat matematiikassa tärkeitä lausekkeita.

\laatikko{Polynomi on lauseke, jossa esiintyy vain:
\begin{itemize}
\item muuttujien potensseja (eksponentti luonnollinen luku, mukaan lukien $0$)
\item kerrottuna jollakin vakiolla sekä
\item näiden potenssitermien summia.
\end{itemize}
}

Polynomissa voi olla yksi tai useampia muuttujia, tai se voi olla muuttujaton vakiopolynomi. Vakiopolynomin tapauksessa muuttujan eksponentin ajatellaan olevan $0$, sillä $x^0=1$, jolloin muuttuja katoaa.

\begin{esimerkki}
Kaikki seuraavat lausekkeet ovat polynomeja:

\begin{tabular}{ll}
$4$ &  ei muuttujia ($4x^0=4 \cdot 1=4$)\\
$2x+1$ &  muuttujana $x$ ($x^1=x$)\\
$5x^2+x-7$ &   muuttujana $x$\\
$-3t^{100}$& muuttujana $t$\\
$y$& muuttujana $y$\\
$y^2+1$& muuttujana $y$\\
$xy^2+x^2y$& muuttujina $x$ ja $y$
\end{tabular}

\end{esimerkki}

\begin{esimerkki}

%itemize?
Miksi muuttujan $t$ lausekkeet 
\[ \frac{2}{t}, \quad \pi^t,  \quad \sqrt{t}+2 \quad \text{ ja } \quad t^\pi \]  \emph{eivät ole} polynomeja?

	\begin{esimratk}

$\frac{2}{t}$ ei ole polynomi, sillä muuttujan $t$ eksponentti ei ole luonnollinen luku: $\frac{2}{t}=2 \cdot \frac{1}{t}= 2t^{-1}$, ja $-1 \notin \mathbb{N}$. \\
$\pi^t$ ei ole polynomi, sillä muuttuja on eksponentissa eikä potenssin kantalukuna. \\
$\sqrt{t}+2$ ei ole polynomi, sillä neliöjuuri ei ole esitettävissä potenssina, jonka eksponentti olisi luonnollinen luku: $\sqrt{t}+2=t^{\frac{1}{2}}+2$, ja $-2 \notin \mathbb{N}$. \\
$\quad t^{\pi}$ ei ole polynomi, sillä muuttujan eksponentti ei ole luonnollinen luku: $\pi \notin \mathbb{N}$. \\
	\end{esimratk}
\end{esimerkki}

%ei kuulu joukkoon -merkki? ^ :S

Polynomi on summalauseke, joiden yhteenlaskettavia kutsutaan \termi{termi}{termeiksi}. Termejä edeltävät miinukset ymmärretään termin osaksi negatiivisina kertoimina. Esimerkiksi polynomin $-2x^3+5x^2+x-7$ termit ovat $-2x^3$, $5x^2$, $x$ ja $-7$. Termejä, jotka eivät sisällä muuttujaa, kutsutaan  \termi{vakiotermi}{vakiotermeiksi}. Vakiotermejä ovat siis esimerkiksi $1$ ja $-7$.

%Yhden muuttujan polynomeissa muuttujana käytetään tyypillisesti kirjainta $x$.
%Yleisimmät muut kirjaimet muuttujien merkitsemiseen ovat $y$ ja $z$.

%etymologiahuomautukset sivun laitaan?

Polynomi-sanan 'poly' on kreikkaa ja tarkoittaa montaa. Erityisesti yhden termin polynomeja kutsutaan \termi{monomi}{monomeiksi}, kahden termin polynomeja \termi{binomi}{binomeiksi}, kolmen termin polynomeja \termi{trinomi}{trinomeiksi} ja niin edelleen:
\begin{center}\begin{tabular}{ccc}
monomi	& binomi 	&	trinomi \\
$-3x^2$   &	$x^2-5$	& $y^2+3y-x$
 \end{tabular} \end{center}

Muuttujan eksponenttia kutsutaan termin \termi{aste}{asteeksi} tai \termi{asteluku}{asteluvuksi}. Vakiotermin aste on nolla. \termi{Polynomin aste}{Polynomin aste} on suurin sen sievennetyn muodon termien asteista. Polynomin asteen saaminen selville voi vaatia polynomin sieventämistä. Esimerkiksi $x^2-x^2+1$ ei ole toisen, vaan nollannen asteen polynomi, sillä sievennettynä $x^2-x^2+1=1$. Useamman muuttujan tapauksessa polynomin termien asteet lasketaan muuttujien potenssien summana.

\begin{esimerkki}
    Mitkä ovat polynomin $x^4-2x^3+5$ termit ja niiden asteet? Mikä on polynomin aste?
    \begin{esimvast}
        Polynomin $x^4-2x^3+5$ termit ovat $x^4$, $-2x^3$ ja $5$ ja niiden asteet ovat
        neljä, kolme ja nolla. Polynomin aste on sama kuin sama kuin
        korkein termien asteista, eli neljä.
    \end{esimvast}
\end{esimerkki}


Koska yhteenlasku on vaihdannainen, polynomin termit voi kirjoittaa missä tahansa järjestyksessä. Esimerkiksi polynomi $y^4+y^2-1$ voidaan kirjoittaa myös järjestyksessä $-1+y^4+y^2$. Yleensä polynomien termit kirjoitetaan niiden asteen perusteella laskevaan järjestykseen niin, että korkeimman asteen termi kirjoitetaan ensin. Useaa muuttujaa sisältävät termit on tapana kirjoittaa ennen yhtä muuttujaa sisältäviä termejä – kuitenkin niin, että potenssijärjestys on tärkeämpi.

\begin{esimerkki}
	Mikä on polynomin $3t^2-1+t^3$ aste? Mikä on korkeimman asteen termin kerroin?
\begin{esimratk}
	
Aloitetaan järjestämällä polynomin termit asteluvun mukaiseen laskevaan järjestykseen: \\
	  $\textcolor{blue}{3t^2}\textcolor{red}{-1}+t^3=$ \\
	  $t^3\textcolor{blue}{+3t^2}\textcolor{red}{-1}$ \\
	
\end{esimratk}

\begin{esimvast}
Polynomi on kolmatta astetta. Korkeimman (eli kolmannen) asteen termin kerroin on $1$.
\end{esimvast}

\end{esimerkki}

\begin{esimerkki}
	Kuinka monetta astetta on (kahden muuttujan) polynomi $x^2+2xy^2+xy$?
\begin{esimratk}
Merkitään selkeyden vuoksi molempien muuttujien kaikkien eksponentit näkyviin: \\
\[x^2+2x^1y^2+x^1y^1\]

Kun termissä on kaksi eri muuttujaa, niin termin asteluku on näiden muuttujien eksponenttien summa. Näin ollen ensimmäisen termin aste on kaksi, toisen termin aste on $1+2=3$ ja kolmannen termin aste on $1+1=2$. Järjestetään termit vielä asteiden mukaiseen laskevaan järjestykseen:

\[\textcolor{blue}{x^2}+\textcolor{red}{2x^1y^2}+x^1y^1\]
\[\textcolor{red}{2x^1y^2}+x^1y^1+\textcolor{blue}{x^2}\]

\end{esimratk} 

\begin{esimvast}
Polynomin asteluku on kolme.
\end{esimvast}

\end{esimerkki}

\subsection*{Polynomifunktion arvo}

\qrlinkki{http://opetus.tv/maa/maa2/polynomiesimerkkeja/}
{Opetus.tv: \emph{polynomiesimerkkejä} (6:59 ja 7:43)}

Polynomi reaalilukujen laskutoimituksena määrittää funktion, jota kutsutaan \termi{polynomifunktio}{polynomifunktioksi}. Oletusarvoisesti kaikkien lukiomatematiikassa käsiteltävien polynomifunktioiden määrittelyjoukko on koko reaalilukujen joukko $\mathbb{R}$, ja tällöin myös funktion arvot reaalilukuja. (Numeeristen ja algebrallisten menetelmien syventävällä kurssilla tarkastellaan polynomifunktioita myös reaalilukujoukkoa laajemmalla kompleksilukualueella $\mathbb{C}$.)

Polynomifunktioita nimetään tyypillisesti suuraakkosin kuten $P$, $Q$ tai $R$.

\begin{esimerkki}

Polynomi $5x^2-3x+2$ määrittää funktion $P:\mathbb{R}\rightarrow \mathbb{R}$, jonka arvot voidaan laskea kaavalla $P(x)=5x^2-3x+2$. Arvoja voidaan laskea siis sijoittamalla joitakin (mielivaltaisia) lukuja muuttujan $x$ paikalle:
\begin{align*}
    P(\textcolor{blue}{2}) & = 5\cdot \textcolor{blue}{2}^2-3\cdot \textcolor{blue}{2}+2 = 20 - 6 + 2 = 16 \\
    P(\textcolor{blue}{-1}) & = 5(\textcolor{blue}{-1})^2-3(\textcolor{blue}{-1})+2 = 5 + 3 + 2 = 10 \\
    P(\textcolor{blue}{-3}) & = 5(\textcolor{blue}{-3})^2-3(\textcolor{blue}{-3})+2 = 45 + 9 + 2 = 56.
\end{align*}
\end{esimerkki}

Polynomeja ja polynomifunktioita käsitellään usein yhtäläisesti; voidaan esimerkiksi sanoa ''polynomi $P(x)=2x+1$'', vaikka tarkoitetaan vastaavaa polynomifunktiota.

%pitäisikö täsmentää? ^

\subsection*{Polynomien yhteen- ja vähennyslasku}

\qrlinkki{http://opetus.tv/maa/maa2/polynomien-yhteen-ja-vahennyslasku/}
{Opetus.tv: \emph{polynomien yhteen- ja vähennyslasku} (7:36)}

Polynomeja voidaan laskea yhteen laskemalla yhteen samanasteiset termit. On kätevää aloittaa ryhmittelemällä samanasteiset termit vierekkäin. Polynomien summat ja erotukset ovat aina polynomeja.

Samanasteisten termien yhteen- ja vähennyslasku perustuu siihen, että reaalilukujen osittelulain (ks. Vapaa matikka 1) nojalla vakiokerroin voidaan ottaa yhteiseksi tekijäksi:
\[ \begin{array}{rcll}
ax^n+bx^n &=&(a+b)x^n, & \text{ esimerkiksi }\\
5x^2+3x^2 &=& (5+3)x^2 &= 8x^2.
\end{array} 
\]

\begin{esimerkki}
Laske polynomien $5x^2-x+5$ ja $3x^2-1$ summa.
    \begin{esimratk}
        \begin{align*}
            (\textcolor{blue}{5x^2} \textcolor{red}{{}-x} + 5) + (\textcolor{blue}{3x^2} -1) 
            &=\textcolor{blue}{5x^2} \textcolor{red}{{}-x} + 5 + \textcolor{blue}{3x^2} -1 \\
            &=\textcolor{blue}{5x^2+3x^2} \textcolor{red}{{}-x} +5-1\\
%                       &=\textcolor{blue}{(5+3)x^2} \textcolor{red}{{}-x}+(5-1)\\
            &=\textcolor{blue}{8x^2} \textcolor{red}{{}-x}+4.
        \end{align*}
    \end{esimratk}
    \begin{esimvast}
        Polynomien summa on $8x^2-x+4$.
    \end{esimvast}
\end{esimerkki}

Polynomeja voidaan vastaavalla tavalla vähentää toisistaan. Kun sulkujen edessä on
miinusmerkki, tulee muistaa vaihtaa termien merkin sulkuja avattaessa.

\begin{esimerkki}
    Laske polynomien $4x^3+1$ ja $6x^3-2$ erotus.
    \begin{esimratk}
        \begin{align*}
		(\textcolor{blue}{4x^3}+1)-(\textcolor{blue}{6x^3}-2) 
		&= \textcolor{blue}{4x^3}+1 - \textcolor{blue}{6x^3} + 2 \\
		&= \textcolor{blue}{4x^3-6x^3} +1 +2 \\
		&=\textcolor{blue}{-2x^3}+3
        \end{align*}
    \end{esimratk}
    \begin{esimvast}
        Polynomien erotus on $-2x^3+3$.
    \end{esimvast}
\end{esimerkki}

\begin{esimerkki}
    Laske polynomien $14x^3+69$ ja $3x^3+2x^2+x$ erotus.
    \begin{esimratk}
        \begin{align*}
            (\textcolor{blue!30!green}{14x^3} + 69) - (\textcolor{blue!30!green}{3x^3} \textcolor{blue}{{}+ 2x^2} \textcolor{red}{{}+x})
            &= \textcolor{blue!30!green}{14x^3} + 69 \textcolor{blue!30!green}{{}-3x^3} \textcolor{blue}{-2x^2} \textcolor{red}{{}-x} \\
            &= \textcolor{blue!30!green}{14x^3{}-3x^3} \textcolor{blue}{{}-2x^2} \textcolor{red}{{}-x} + 69 \\
%           &= \textcolor{blue!30!green}{(14{}-3)x^3} \textcolor{blue}{{}-2x^2} \textcolor{red}{{}-x} + 69 \\
            &= \textcolor{blue!30!green}{11x^3} \textcolor{blue}{{}-2x^2} \textcolor{red}{{}-x} + 69
        \end{align*}
    \end{esimratk}
    \begin{esimvast}
        Polynomien erotus on $11x^3-2x^2-x+69$.
    \end{esimvast}
\end{esimerkki}

\begin{esimerkki}
    Olkoot polynomit $P(x)=2x+1$ ja $Q(x)=3x^2-2x+5$. Määritä summa $R(x)=P(x)+Q(x)$.
    \begin{esimratk}
        \begin{align*}
            R(x) = P(x)+Q(x) &= (2x+1)+(3x^2-2x+5) \\
                             &= 2x+1+3x^2-2x+5 \\
                             &= 3x^2+2x-2x+1+5 \\
                             &= 3x^2+6.
        \end{align*}
    \end{esimratk}
    \begin{esimvast}
        $R(x) = 3x^2+6$.
    \end{esimvast}
\end{esimerkki}

%Polynomit sievennetään yleensä yleiseen muotoon, jossa on vain yksi termi kutakin astetta kohti.
%Tämä tehdään esimerkiksi silloin, kun selvitetään polynomin aste.

\begin{esimerkki}
    Laske polynomien $P$ ja $Q$ erotus $R$, kun $P(x)=-3x^4+x^2+1$ ja $Q(x)=-3x^4+3x^3-x$.
    Mikä on polynomin $R$ aste?
   \begin{esimratk}
        \begin{align*}
            R(x) = P(x)-Q(x) &= (-3x^4+x^2+1)-(-3x^4+3x^3-x) \\
                             &= -3x^4+x^2+1+3x^4-3x^3+x \\
                             &= -3x^4+3x^4-3x^3+x^2+x+1 \\
                             &= -3x^3+x^2+x+1.
        \end{align*}
    \end{esimratk}
    \begin{esimvast}
        $R(x) = -3x^3+x^2+x+1$. Polynomin $R$ aste on kolme.
    \end{esimvast}
\end{esimerkki}

\subsection*{Polynomin yleinen muoto}

Formaalisti kirjoitettuna yhden muuttujan polynomin yleinen muoto on
\[
a_n x^n + a_{n-1} x^{n-1} + \ldots + a_1 x + a_0 \] 
jollakin $n\in\N$. Muuttuja $n$ kertoo polynomin asteen, ja kertoimet $a_0, a_1, \ldots, a_{n-1}, a_n$ ovat reaalilukuvakioita. Jos jokin kertoimista on $0$, kyseinen termi katoaa.

\begin{esimerkki}

Polynomi $-3x^13-\frac{3}{4}x^2-17$ on polynomin yleistä muotoa siten, että $n=13$, $a_{13}=-3$, $a_2=-\frac{3}{4}$ ja $a_0=-17$, ja kaikki muut kertoimet ovat nollia.

\end{esimerkki}
 
Jokainen lauseke, joka on esitettävissä polynomin yleisessä muodossa, on polynomi.

\begin{esimerkki}

Lauseke $\frac{2x^2+1}{x}$ ei ensi näkemältä näytä polynomilta murtolausekkeen nimittäjässä olevan $x$:n vuoksi. Lauseketta on kuitenkin mahdollista sieventää esimerkiksi seuraavasti: \\


$\frac{2x^3+x^2}{x} = \frac{2x^3}{x}+\frac{x^2}{x} = 2x^2+x $ \\

Kyseessä oli siis kuitenkin toisen asteen kaksiterminen yhden muuttujan polynomi. (Huomataan kuitenkin, että alkuperäisen esitysmuodon vuoksi $x \neq 0$, koska nollalla jakamista ei ole määritelty.)

\end{esimerkki}

\begin{tehtavasivu}

\paragraph*{Opi perusteet}

\begin{tehtava}
    Mitkä seuraavista ovat polynomeja?
    \begin{enumerate}[a)]
        \item $\frac{1}{x}$
       %\item $x^3+4x$
        \item $5x-125$
       %\item $2^x$
        \item $\sqrt{x}+1$
        \item $3x^4+6x^2+9$
        \item $\sqrt{2}x-x$
        \item $4^x+5x+6$
    \end{enumerate}
    \begin{vastaus}
        \begin{enumerate}[a)]
            \item Ei ole.
           %\item On.
            \item On.
           %\item Ei ole.
            \item Ei ole.
            \item On.
            \item On.
            \item Ei ole.
        \end{enumerate}
    \end{vastaus}
\end{tehtava}

\begin{tehtava}
	Mikä on/mitkä ovat polynomin \\ $P(x) = x^5-3x^3+2x-1$
	\begin{enumerate}[a)]
		\item aste
		\item termit
		\item kolmannen asteen termi
		\item kolmannen termin aste
		\item vakiotermi
	\end{enumerate}

	\begin{vastaus}
		\begin{enumerate}[a)]
			\item $5$
			\item $x^5$, $-3x^3$, $2x$, $-1$
			\item $-3x^3$
			\item $1$
			\item $-1$
		\end{enumerate}
	\end{vastaus}
\end{tehtava}


\begin{tehtava}
    Täydennä taulukko. Polynomeissa on vain yksi muuttuja, $x$.
        
    \begin{tabular}{|c|c|c|c|c|}
                                                                         \hline
polynomi     & \begin{sideways}\begin{minipage}{3.5cm}termien\\lukumäärä\end{minipage}\end{sideways}%
& \begin{sideways}\begin{minipage}{3.5cm}korkeimman asteen\\termin kerroin\end{minipage}\end{sideways}%
& \begin{sideways}\begin{minipage}{3.5cm}polynomin\\asteluku\end{minipage}\end{sideways}%
& \begin{sideways}vakiotermi\end{sideways} \\ \hline
$-2x^2+6x$   &        2  &         $-2$      &       2   &    0       \\ \hline 
$7x^3-x-15$  &           &                   &           &            \\ \hline 
             &        2  &          $-9$     &       2   &    5       \\ \hline 
%             &        3  &          $-1$     &       5   &    $-17$   \\ \hline 
%             &        4  &                   &       3   &            \\ \hline 
             &        1  &          -5       &       99  &            \\ \hline                           
    \end{tabular}

%      \begin{tabular}{|l|c|c|c|c|c|c|}
%                                                                                            \hline
% polynomi     & \begin{sideways}$-2x^2+6x$\end{sideways} & \begin{sideways}$7x^3-x-15$\end{sideways}    &     &          &     &     \\ \hline
% termien      &            &                &     &          &     &     \\ \hline 
% lukumäärä    &        2   &                & 2   &    3     &  4  &  1  \\ \hline 
% korkeimman & & & & & & \\  
% asteen & & & & & & \\  
% termin & & & & & & \\  
% kerroin      &    $-2$    &                &$-9$ &   $-1$   &     &$-5$ \\ \hline 
% polynomin & & & & & & \\  
% asteluku     &        2   &                & 2   &    5     &  3  & 99  \\ \hline 
% vakiotermi   &        0   &                & 5   &    $-17$ &     &     \\ \hline 
%     \end{tabular}
%      \begin{tabular}{|c|c|c|c|c|}
%                                                                                           \hline
%              & termien   & korkeimman asteen & polynomin &            \\
% polynomi     & lukumäärä & termin kerroin    & asteluku  & vakiotermi \\ \hline
% $-2x^2+6x$   &        2  &         $-2$      &       2   &    0       \\ \hline 
% $7x^3-x-15$  &           &                   &           &            \\ \hline 
%              &        2  &          $-9$     &       2   &    5       \\ \hline 
%              &        3  &          $-1$     &       5   &    $-17$   \\ \hline 
%              &        4  &                   &       3   &            \\ \hline 
%              &        1  &          -5       &       99  &            \\ \hline                           
%     \end{tabular}

    
    \begin{vastaus}

	    \begin{tabular}{|c|c|c|c|c|}
                                                                                           \hline
polynomi     & \begin{sideways}\begin{minipage}{3.5cm}termien\\lukumäärä\end{minipage}\end{sideways}%
& \begin{sideways}\begin{minipage}{3.5cm}korkeimman asteen\\termin kerroin\end{minipage}\end{sideways}%
& \begin{sideways}\begin{minipage}{3.5cm}polynomin\\asteluku\end{minipage}\end{sideways}%
& \begin{sideways}vakiotermi\end{sideways} \\ \hline
$-2x^2+6x$   &        2          &         $-2$      &       2             &    0       \\ \hline 
$7x^3-x-15$  &        3          &           7       &       3             &    $-15$   \\ \hline 
$-9x^2+5$    &        2          &          $-9$     &       2             &    5       \\ \hline 
%$-x^5\textcolor{blue}{+4x}-17$%
%             &        3          &          $-1$     &       5             &    $-17$   \\ \hline 
%$\textcolor{blue}{8}x^3\textcolor{blue}{-x^2+4x}-17$%
%             &        4          &\textcolor{blue}{8}  &       3             &\textcolor{blue}{17}\\ \hline 
$-5x^{99}$   &        1          &          $-5$     &       99            &         0      \\ \hline                           
   	  \end{tabular}
     \end{vastaus}
\end{tehtava}

\begin{tehtava}
    Sievennä.
    \begin{enumerate}[a)]
        \item $3x+5x $
        \item $4x^2+7x^2$
        \item $-6y+2y $
        \item $3x-(-2x)$
    \end{enumerate}
    \begin{vastaus}
        \begin{enumerate}[a)]
            \item $8x$
            \item $11x^2$
            \item $-4y$
            \item $5x$
        \end{enumerate}
    \end{vastaus}
\end{tehtava}

\begin{tehtava}
    Sievennä.
    \begin{enumerate}[a)]
    	\item $5x-2+2x+7$
        \item $5x-3y-y-2x$
        \item $2x^2+x+x^2-5x$
        \item $y^3 - 2y^2+4y^3-y $
    \end{enumerate}
    \begin{vastaus}
        \begin{enumerate}[a)]
        	\item $7x+5$
            \item $3x-4y$
            \item $3x^2-4x$
            \item $5y^3-2y^2-y$
        \end{enumerate}
    \end{vastaus}
\end{tehtava}

\begin{tehtava}
    Olkoot $P(x)=x^2+5$ ja $Q(x)=x^3-1$. Laske
    \begin{enumerate}[a)]
        \item polynomin $P(x)$ arvo, kun $x=2$
        \item polynomin $Q(x)$ arvo, kun $x=1$
        \item $P(-7)$
        \item $Q(-1)$.
    \end{enumerate}
    \begin{vastaus}
        \begin{enumerate}[a)]
            \item $9$ % 2^2 + 5 = 4 + 5 
            \item $0$ % 1^3 - 1
            \item $54$ % (-7)^2 + 5 = 49 + 5 
            \item $-2$ % (-1)^3 - 1 = -1 -1
        \end{enumerate}
    \end{vastaus}
\end{tehtava}

\paragraph*{Hallitse kokonaisuus}

\begin{tehtava}
    Sievennä.
    \begin{enumerate}[a)]
        \item $(x^2 - 2x + 1) + (-x^2 + x) $
        \item $(3y^3 + 2y^2  + y) - (-y^2 + y)$
        \item $(z^{10} - z^6 + z^2 + 1) + (z^{10} + 2z^8 - 3z^6)$
    \end{enumerate}
    \begin{vastaus}
        \begin{enumerate}[a)]
            \item $-x + 1$
            \item $3y^3 + 3y^2$
            \item $2z^{10} + 2z^8 - 4z^6 + z^2 + 1$
        \end{enumerate}
    \end{vastaus}
\end{tehtava}

\begin{tehtava}
	Mitkä ovat seuraavien polynomifunktioiden asteet, ts. sievennettyjen muotojen asteet?
	\begin{enumerate}[a)]
		\item $x+5-x$
		\item $x^2+x-2x^2$
		\item $4x^5+x^2-4-x^2$
		\item $x^4-2x^3+x-1+x^3-x^4+x^3$
	\end{enumerate}

	\begin{vastaus}
		\begin{enumerate}[a)]
			\item $0$
			\item $2$
			\item $5$
			\item $1$
		\end{enumerate}
	\end{vastaus}
\end{tehtava}

\begin{tehtava}
	Mitkä seuraavista polynomilausekkeista esittävät samaa polynomifunktiota kuin
	$x^3+2x+1$?
	\begin{enumerate}[a)]
		\item $2x+x^3+1$
		\item $x^2+2x+1$
		\item $x+2x^3+1 - (x^3+x)$
		\item $15+x^4+2x+x^3-x^4-14$
	\end{enumerate}
	\begin{vastaus}
		a) ja d)
	\end{vastaus}
\end{tehtava}

\begin{tehtava}
    Olkoot $P(x)=x^2+3x+4$ ja $Q(x)=x^3-10x+1$. Sievennä
    \begin{enumerate}[a)]
        \item $P(x)+Q(x)$
        \item $P(x)-Q(x)$
        \item $Q(x)-P(x)$
        \item $2P(3)+Q(2)$.
    \end{enumerate}
    \begin{vastaus}
        \begin{enumerate}[a)]
            \item $x^3+x^2-7x+5$ % x^2+3x+4 + x^3-10x+1
            \item $-x^3+x^2+13x+3$ % x^2+3x+4 -(x^3-10x+1) = x^2+3x+4 -x^3+10x-1
            \item $x^3-x^2-13x-3$ % 
            \item $33$ % 2*(3^2+3*3+4) +  2^3-10*2+1 = 2*(9+9+4)+8-20+1 =44-11 =33
        \end{enumerate}
    \end{vastaus}
\end{tehtava}

\begin{tehtava}
	Sievennä polynomifunktiot ja laske funktioiden arvot muuttujan arvoilla $1$, $-1$ ja $3$.
	\begin{enumerate}[a)]
		\item $P(x)=(x^3-4x+5)+(-x^3+x^2+4x-2)$
		\item $Q(x)=(2x^3+x^2-10x)-(3x^3-4x^2+5x)$
	\end{enumerate}
	
	\begin{vastaus}
		\begin{enumerate}[a)]
			\item $P(x)=x^2+3$, $P(1)=4$, $P(-1)=4$ ja $P(3)=12$
			\item $Q(x)=-x^3+5x^2-15x$, $Q(1)=21$, $Q(-1)=-9$ ja $R(3)=-27$
		\end{enumerate}
	\end{vastaus}
\end{tehtava}

\begin{tehtava}
	Hyödynnä edellisen tehtävän polynomifunktioita ja laske
	\begin{enumerate}[a)]
		 \item polynomien $P(x)$ ja $Q(x)$ summa
		 \item lukujen $P(1)$ ja $Q(-2)$ erotus
	\end{enumerate}
	
	\begin{vastaus}
		\begin{enumerate}[a)]
			\item $-x^3+6x^2-15x+3$
			\item $58$
	\end{enumerate}
	\end{vastaus}
\end{tehtava}


\paragraph*{Lisää tehtäviä}

\begin{tehtava}
    Olkoot $P(x)=x^2+3x+4$ ja \\ $Q(x)=x^3-10x+1$. Laske:
    \begin{enumerate}[a)]
        \item $P(-1)$
        \item $Q(-2)$
        \item $P(3)$
        \item $Q(0)$
    \end{enumerate}
    \begin{vastaus}
        \begin{enumerate}[a)]
            \item $2$
            \item $13$
            \item $22$
            \item $1$
        \end{enumerate}
    \end{vastaus}
\end{tehtava}


\begin{tehtava}
	Mitkä ovat seuraavien polynomien asteet?
	\begin{enumerate}[a)]
		\item $x^2 + 3x - 5$
		\item $100 + x$
		\item $3x^3 + 90x^8 + 2x$
		\item $12x^1 + 34x^2 + 56x^3 + 78x^5 + 90x^5$
	\end{enumerate}

	\begin{vastaus}
		\begin{enumerate}[a)]
			\item $2$
			\item $1$
			\item $8$
			\item $5$
		\end{enumerate}
	\end{vastaus}
\end{tehtava}

\begin{tehtava}
     Sievennä
     \begin{enumerate}[a)]
         \item $(2x + 3) + x $
         \item $(3x - 1) + (-x + 1)$
         \item $(5x + 10) + (6x - 6) - (x + 3)$
     \end{enumerate}
     \begin{vastaus}
         \begin{enumerate}
             \item $3x + 3$
             \item $2x$
             \item $10x + 1$
         \end{enumerate}
     \end{vastaus}
 \end{tehtava}

\begin{tehtava}
	Sievennä polynomifunktiot ja laske funktioiden arvot muuttujan arvoilla $1$, $-1$ ja $3$.
	\begin{enumerate}[a)]
		\item $R(x)=4(2x-4)+(-x^3+1)$
		\item $S(x)=-(2x^2-x+8)+6(x^5-3x^2+1)$ \\ $-2(3x^5-10x^2)$
	\end{enumerate}
	\begin{vastaus}
		\begin{enumerate}[a)]
			\item $R(x)=-x^3+8x-15$, $R(1)=-8$, $R(-1)=-22$ ja $R(3)=-18$
			\item $S(x)=-2$, $S(1)=-1$, $S(-1)=-3$ ja $S(3)=1$
		\end{enumerate}
	\end{vastaus}
\end{tehtava}

\begin{tehtava}
	$\star$ Määritellään kahden reaalimuuttujan polynomifunktio kaavalla $f(x,y)=xy^2+x^2y$.
		\begin{enumerate}
		\item Laske funktion arvo $f(-1,2)$.
		\item Onko $f(x,y)=f(y,x)$ kaikilla $x$ ja $y$?
		\end{enumerate}	
	
	\begin{vastaus}
		\begin{enumerate}[a)]
			\item $f(-1,2)=(-1)\cdot 2^2+(-1)^2 \cdot 2=-4+2=-2$
			\item Kyllä.
		\end{enumerate}
	\end{vastaus}
\end{tehtava}

\begin{tehtava}
	$\star$ Kahden muuttujan ($x$ ja $y$) binomista tiedetään, että sen asteluku on kaksi, vakiotermejä ei ole, ja kaikkien termien kertoimet ovat ykkösiä. Luettele kaikki mahdolliset polynomit, jotka toteuttavat nämä ehdot.
	
	\begin{vastaus}
		$x^2+y$, $x^2+xy$, $y^2+x$, $y^2+xy$, $x^2+y^2$
	\end{vastaus}
\end{tehtava}

\end{tehtavasivu}