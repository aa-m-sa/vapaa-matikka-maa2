\section{$\star$ Korkeamman asteen polynomifunktio}

\subsubsection*{Joitakin yleistyksiä}

Polynomeille voidaan todistaa seuraavat tulokset:

\laatikko{
\begin{itemize}
\item Kaikki polynomit voidaan jakaa tekijöihin, jotka ovat korkeintaan toista astetta.
\item Paritonasteisilla polynomeilla on vähintään yksi nollakohta.
\item Asteen $n$ polynomilla on korkeintaan $n$ nollakohtaa.
\end{itemize}}

Todistetaan näistä viimeinen:

\begin{todistus}
Jos $n$ asteen polynomilla $P$ olisi yli $n$ nollakohtaa, sillä olisi yli $n$ ensimmäisen asteen
tekijää. Polynomin $P$ aste olisi siis yli $n$, mikä on ristiriita. Nollakohtia on siis
korkeintaan $n$.
\end{todistus}

Tämä tulos kertoo paljon siitä, millaisia polynomien kuvaajat voivat olla.


\begin{esimerkki} Kolmannen asteen polynomilla on 1--3 nollakohtaa.

\begin{lukusuora}{-2.5}{3}{3.6}
\lukusuoraisobbox
\lukusuorakuvaaja{(x**3-x-1)/2}
\lukusuorapienipiste{1.32472}{}
\end{lukusuora}
\begin{lukusuora}{-2.8}{2.5}{3.6}
\lukusuoraisobbox
\lukusuorakuvaaja{(x**3-x+0.3849)/2}
\lukusuorapienipiste{-1.1547}{}
\lukusuorapienipiste{0.577028}{}
\end{lukusuora}
\begin{lukusuora}{-2}{2}{3.6}
\lukusuoraisobbox
\lukusuorakuvaaja{1.4*(x**3-x)}
\lukusuorapienipiste{-1}{}
\lukusuorapienipiste{0}{}
\lukusuorapienipiste{1}{}
\end{lukusuora}

\end{esimerkki}


\begin{esimerkki} Neljännen asteen polynomilla on 0--4 nollakohtaa.

\begin{lukusuora}{-4}{4}{3.6}
\lukusuorabboxy{-0.5}{1.5}
\lukusuorakuvaaja{(x**4-5*x**2+12)/14}
\end{lukusuora}
\begin{lukusuora}{-4}{4}{3.6}
\lukusuorabboxy{-0.5}{1.5}
\lukusuorakuvaaja{(x**4-5*x**2+3*x+11.2)/14}
\lukusuorapienipiste{-1.71394}{}
\end{lukusuora}
\begin{lukusuora}{-4}{4}{3.6}
\lukusuorabboxy{-0.5}{1.5}
\lukusuorakuvaaja{(x**4-5*x**2-3)/14}
\lukusuorapienipiste{2.354}{}
\lukusuorapienipiste{-2.354}{}
\end{lukusuora}

\begin{lukusuora}{-4}{4}{3.6}
\lukusuorabboxy{-0.5}{1.5}
\lukusuorakuvaaja{(x**4-5*x**2+3*x+1.75842)/14}
\lukusuorapienipiste{-2.43622}{}
\lukusuorapienipiste{-0.367327}{}
\lukusuorapienipiste{1.402}{}
\end{lukusuora}
\begin{lukusuora}{-2.8}{2.8}{3.6}
\lukusuorabboxy{-0.5}{1.5}
\lukusuorakuvaaja{0.6*(x+1.5)*(x+0.5)*(x-0.5)*(x-1.5)}
\lukusuorapienipiste{1.5}{}
\lukusuorapienipiste{0.5}{}
\lukusuorapienipiste{-1.5}{}
\lukusuorapienipiste{-0.5}{}
\end{lukusuora}

\end{esimerkki}

\begin{tehtavasivu}

\begin{tehtava}
    Anna esimerkki
    \begin{alakohdat}
	\alakohta{kolmannen asteen polynomista, jolla on kaksi nollakohtaa}
	\alakohta{neljännen asteen polynomista, jolla on yksi nollakohta}
	\alakohta{neljänen asteen polynomista, jolla ei ole nollakohtia}
\end{alakohdat}
    \begin{vastaus}
	Esimerkiksi    
    \begin{alakohdat}
	\alakohta{$P(x)=x^2(x-1)$ (nollakohdat $x= 0$ ja $x = 1$)}
	\alakohta{$Q(x)=x^4$ (nollakohta $x=0$)}
	\alakohta{$R(x)=x^4+1$}
\end{alakohdat}
    \end{vastaus}
\end{tehtava}

\begin{tehtava}
    Kolmannen asteen polynomille $P$ pätee $P(1)=P(2)=P(3)=-2$ ja $P(0)=16$. Ratkaise $P(x)$.
    \begin{vastaus}
        $P(x)=-3(x-1)(x-2)(x-3)-2=-3x^3+18x^2-33x+18$
    \end{vastaus}
\end{tehtava}

\end{tehtavasivu}