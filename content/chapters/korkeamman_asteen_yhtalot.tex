\section{Korkeamman asteen yhtälöt}

\qrlinkki{http://opetus.tv/maa/maa2/n-asteinen-polynomifunktio/}{Opetus.tv: \emph{N-asteinen polynomifunktio} (10:40)}

\qrlinkki{http://opetus.tv/maa/maa2/n-asteinen-polynomiyhtalo/}{Opetus.tv: \emph{N-asteinen polynomiyhtälö} (8:38, 6:20 ja 15:53)}

Jos toisen asteen polynomiyhtälöllä on reaalilukuratkaisuja, ne löytyvät ratkaisukaavalla.
Myös kolmannen ja neljännen asteen yhtälöille on olemassa ratkaisukaavat.
Ne ovat kuitenkin niin monimutkaisia, että ne eivät juuri sovellu käsin laskettaviksi, eikä niitä siksi esitellä tässä.
Korkeamman kuin neljännen asteen yhtälöille sen sijaan ei edes ole olemassa yleistä ratkaisukaavaa.
Tämän osoitti norjalainen matemaatikko Niels Henrik Abel vuonna 1823.

Korkeamman kuin toisen asteen yhtälöt ratkaistaan käytännössä yleensä tietokoneen avulla.
Niille yhtälöille, joille ei ole ratkaisukaavaa, ratkaiseminen onnistuu vain numeerisesti eli likiarvoja käyttäen.
Joissain erikoistapauksissa korkeamman asteen yhtälön ratkaiseminen onnistuu myös käsin.
Seuraavassa tarkastellaan tällaisia erikoistapauksia.

\subsection*{Tekijöihinjako}

Polynomiyhtälöitä voidaan toisinaan ratkaista jakamalla polynomi tekijöihin.

%Jos polynomiyhtälössä $P(x) = 0$ polynomi $P(x)$ voidaan jakaa tekijöihin, ratkaisu saadaan etsimällä näiden tekijöiden nollakohdat.
%Esimerkiksi vakiotermittömässä polynomissa voidaan ottaa muuttuja yhteiseksi tekijäksi ja riittää ratkaista yhtä pienemmän asteen %polynomiyhtälö.

\begin{esimerkki}
Ratkaise yhtälö $x^3 - 3x^2 + x = 0$.

\begin{esimratk}
Polynomissa $x^3 - 3x^2 + x$ ei ole vakiotermiä. Voidaan siis ottaa yhteiseksi tekijäksi $x$, jolloin polynomi tulee muotoon $x(x^2 - 3x + 1)$. 

Nyt yhtälön ratkaiseminen voidaan aloittaa soveltamalla tulon nollasääntöä:
\begin{align*}
x^3 - 3x^2 + x & =0 \\
x(x^2 - 3x + 1) & =0 \\
x=0 \quad & \text{tai} \quad x^2 - 3x + 1 = 0 \\
\end{align*}

Jäljelle jäävään toisen asteen yhtälöön $x^2 - 3x + 1 = 0$ voidaan käyttää ratkaisukaavaa:
\[
x =\frac{3\pm\sqrt{3^2-4\cdot 1\cdot 1}}{2\cdot 1}=\frac{3\pm \sqrt{5}}{2}.
\]
\end{esimratk}

\begin{esimvast} $x=0$ tai $x=\dfrac{3\pm \sqrt{5}}{2}$
\end{esimvast}
\end{esimerkki}

Jos polynomi voidaan jakaa tekijöihin, sen ratkaiseminen helpottuu, koska voidaan soveltaa tulon nollasääntöä.
Sopiva tekijöihinjako on kuitenkin usein vaikea löytää. Tarkastellaan vielä toista esimerkkiä.
%Aiemmin esitellyn polynomien jakolauseen mukaan kaikki polynomit voidaan jakaa tekijöihin, jotka ovat korkeintaan toista astetta.
%Periaatteessa tällä tavalla voidaan siis ratkaista kaikki polynomiyhtälöt. Tekijöihin jakaminen on kuitenkin yleensä vaikeaa.

\begin{esimerkki}
Ratkaise yhtälö $x^3-17x^2-x+17 = 0$.

\begin{esimratk}
Yhtälön vasemmalla puolella olevan polynomin voi jakaa tekijöihin ryhmittelemällä:

\begin{align*}
x^3-17x^2-x+17=x^2(x-17)+(-1)(x-17)=(x^2-1)(x-17).
\end{align*}

Nyt yhtälö ratkeaa tulon nollasäännöllä:
\begin{align*}
x^3-17x^2&-x+17=0 \\
(x^2-1)&(x-17)=0 \\
x^2-1 = 0 \quad &\text{tai} \quad x - 17 = 0 \\
x^2 = 1 \quad &\text{tai} \quad x = 17 \\
x =\pm 1 \quad &\text{tai} \quad x = 17 \\
\end{align*}
\end{esimratk}

\begin{esimvast}
$x = 17$, $x = 1$ tai $x=-1$
\end{esimvast}
\end{esimerkki}

\subsection*{Sijoitukset}

Joskus yhtälöt ratkeavat, kun niihin sijoitetaan jokin apumuuttuja.
Tällöin puhutaan myös muuttujanvaihdosta.

%Esimerkiksi muotoa $ax^4+bx^2+c=0$ olevissa yhtälöissä huomataan, että merkitsemällä lauseketta $x^2$ kirjaimella $y$, yhtälö voidaan kirjoittaa muotoon $ay^2+by+c=0$. Uudesta yhtälöstä voidaan ratkaista $y$ toisen asteen yhtälön ratkaisukaavalla, ja sijoituksesta $y = x^2$ voidaan ratkaista $x$.

\begin{esimerkki}
Ratkaise yhtälö $2x^4+14x^2-36=0$.

\begin{esimratk}
Koska $x^4=(x^2)^2$, ratkaistava yhtälö voidaan kirjoittaa myös muodossa $2(x^2)^2+14x^2-36=0$.
Kun nyt sijoitetaan lausekkeen $x^2$ paikalle $y$, eli merkitään $y=x^2$, saadaankin muuttujan $y$ yhtälö
\[
2y^2+14y-36=0.
\]
Tämä on toisen asteen yhtälö, joka osataan ratkaista esimerkiksi ratkaisukaavalla.
Tällä tavoin saadaan
\[
y=\frac{-14\pm\sqrt{14^2-4\cdot 2\cdot(-36)}}{2\cdot 2}=\frac{-14\pm 22}{4}.
\]
Ratkaisut ovat siis $y=2$ ja $y=-9$.

On kuitenkin vielä selvitettävä alkuperäisen muuttujan $x$ arvot.
Koska $x^2=y$, saadaan $y$:lle löydetyistä arvoista yhtälöt $x^2=2$ ja $x^2=-9$.
Reaaliluvun neliö ei kuitenkaan voi olla negatiivinen, joten ainoat ratkaisut ovat yhtälön $x^2 = 2$ ratkaisut.
Ne ovat $x=\pm\sqrt{2}$.
\end{esimratk}

\begin{esimvast}
$x=2$ tai $x=-2$
\end{esimvast}
\end{esimerkki}

Muotoa $ax^4+bx^2+c=0$ oleva yhtälö (eli ns. bikvadraattinen yhtälö) voidaan aina ratkaista sijoittamalla $y=x^2$.
Yleisemmin muotoa $ax^{2n}+bx^n+c=0$ olevat yhtälöt voidaan ratkaista sijoituksella $y = x^n$.

\begin{esimerkki}
Ratkaise yhtälö $x^{10}+x^5=2$.

\begin{esimratk}
Muutetaan yhtälö muotoon $(x^5)^2+x^5-2=0$ ja tehdään sijoitus $y = x^5$.
Nyt yhtälö saa muodon $y^2+y-2 = 0$.
Toisen asteen yhtälön ratkaisukaavalla saadaan yhtälön ratkaisuiksi $y = -2$ ja $y = 1$.

Nyt alkuperäisen yhtälön ratkaisut saadaan yhtälöistä $x^5=-2$ ja $x^5=1$. Siten ratkaisut ovat $x = \sqrt[5]{-2}$ ja $x = 1$.
\end{esimratk}

\begin{esimvast}
$x = \sqrt[5]{-2}$ tai $x = 1$
\end{esimvast}

\end{esimerkki}

\begin{tehtavasivu}

\paragraph*{Opi perusteet}

\begin{tehtava}
    Ratkaise yhtälöt.
    \begin{alakohdat}
        \alakohta{$x^3-5x^2+6x=0$}
        \alakohta{$x^4 - 16 = 0$}
        \alakohta{$x^6 - x^4 = 0$}
    \end{alakohdat}
    \begin{vastaus}
        \begin{alakohdat}
            \alakohta{$x = 0$ tai $x=2$ tai $x=3$}
            \alakohta{$x = \pm 2$}
            \alakohta{$x = 0$ tai $x=\pm 1$}
        \end{alakohdat}
    \end{vastaus}
\end{tehtava}

\begin{tehtava}
    Ratkaise yhtälöt.
    \begin{alakohdat}
        \alakohta{$x^4 - 2x^2 - 24 = 0$}
        \alakohta{$x^4 - 4x^2 - 5 = 0$}
        \alakohta{$x^4 - 8x^2 + 15 = 0$}
    \end{alakohdat}
    \begin{vastaus}
        \begin{alakohdat}
            \alakohta{$x = \pm\sqrt{6}$}
            \alakohta{$x = \pm\sqrt{5}$}
            \alakohta{$x = \pm\sqrt{3}$ tai $\pm\sqrt{5}$}
        \end{alakohdat}
    \end{vastaus}
\end{tehtava}

\begin{tehtava}
    Ratkaise yhtälöt.
    \begin{alakohdat}
        \alakohta{$x^8 - 1 = 0$}
        \alakohta{$x^8 - x^4 = 0$}
        \alakohta{$x^8 - x^4 - 1 = 0$}
    \end{alakohdat}
    \begin{vastaus}
        \begin{alakohdat}
            \alakohta{$x = \pm\sqrt{1}$}
            \alakohta{$x = 0$ tai $x = \pm\sqrt{1}$}
            \alakohta{$x = \pm\sqrt[4]{\frac{1+\sqrt{5}}{2}} = \pm\sqrt[4]{\upvarphi}$ ($\upvarphi$ on kultaisena leikkauksena tunnettu vakio)}
        \end{alakohdat}
    \end{vastaus}
\end{tehtava}

\paragraph*{Hallitse kokonaisuus}

\begin{tehtava}
Ratkaise yhtälöt.
\begin{alakohdat}
\alakohta{$x^5-3x^4+2x^3=0$}
\alakohta{$x^4+5x^3-x^2-5x=0$}
\alakohta{$x^3-4x^2-4x+16=0$}
\end{alakohdat}
\begin{vastaus}
\begin{alakohdat}
\alakohta{$x=0$ tai $x=1$ tai $x=2$}
\alakohta{$x=0$ tai $x=-5$ tai $x= \pm 1$}
\alakohta{$x=4$ tai $x= \pm 2$}
\end{alakohdat}
\end{vastaus}
\end{tehtava}

\begin{tehtava}
    Ratkaise yhtälöt.
    \begin{alakohdat}
        \alakohta{$x^4 - 16 = 0$}
        \alakohta{$2x^4 = 8x^2$}
        \alakohta{$x^6 - 2x^3 = 3$}
        \alakohta{$x^{100} - 2x^{50} + 1 = 0$}
    \end{alakohdat}
    \begin{vastaus}
        \begin{alakohdat}
            \alakohta{$x = \pm2$}
            \alakohta{$x = 0$ tai $x=\pm2$}
            \alakohta{$x = \sqrt[3]{3}$ tai $x= -1$}
            \alakohta{$x = \pm1$}
        \end{alakohdat}
    \end{vastaus}
\end{tehtava}

\begin{tehtava}
	Ratkaise yhtälö $x^{627} - 6x^{514} + 5x^{401} = 0$.
	\begin{vastaus}
		$x = 0$, $x = 1$ tai $x = \sqrt[113]{5}$
	\end{vastaus}
\end{tehtava}


\begin{tehtava}
 	Ratkaise yhtälö $5^{x^3+4x^2+x}=1$. 
%	(K02/T2b) Ratkaise yhtälö $e^{x^3+4x^2+x}=1$. [$e$ on matemaattinen vakio, irrationaaliluku, jonka likiarvo on $2,718$.]
	\begin{vastaus}
	$x=0$ tai $x=-2 + \sqrt[]{3}$ tai $x=-2 - \sqrt[]{3}$
	\end{vastaus}
\end{tehtava}

\begin{tehtava}
	$ \star $ Ratkaise yhtälö $(x+\frac{1}{x})^2-x-\frac{1}{x}-6 = 0$.
	\begin{vastaus}
		$x = -1$, $x = \frac{3\pm \sqrt{5}}{2}$
	\end{vastaus}
\end{tehtava}

\begin{tehtava}
	$ \star $ Ratkaise yhtälö $2^x-1=\frac{12}{2^x}$
	\begin{vastaus}
	$x=2$
	\end{vastaus}
\end{tehtava}

\paragraph*{Lisää tehtäviä}

\begin{tehtava}
    Ratkaise yhtälöt.
    \begin{alakohdat}
        \alakohta{$x^4 + 7x^3 = 0$}
        \alakohta{$2x^3 - 16x^2 + 32x = 0$}
        \alakohta{$x^6 + 6x^5 = -9x^4$}
        \alakohta{$x^3 - 2x^5 = 0$}
    \end{alakohdat}
    \begin{vastaus}
        \begin{alakohdat}
        	\alakohta{$x = 0$ tai $x = -7$}
        	\alakohta{$x = 0$ tai $x = 4$}
        	\alakohta{$x = 0$ tai $x = -3$}
            \alakohta{$x = 0$ tai $x = \pm\dfrac{1}{\sqrt{2}}$}
        \end{alakohdat}
    \end{vastaus}
\end{tehtava}

\begin{tehtava}
    Ratkaise yhtälöt.
    \begin{alakohdat}
        \alakohta{$x^5 - 2x^3 + x = 0$}
        \alakohta{$x^8 + 4x^4 = 5x^6$}
    \end{alakohdat}
    \begin{vastaus}
        \begin{alakohdat}
        	\alakohta{$x = 0$ tai $x = \pm1$}
        	\alakohta{$x = 0$ tai $x = \pm1$ tai $x = \pm2$}
        \end{alakohdat}
    \end{vastaus}
\end{tehtava}

\begin{tehtava} % Korkeamman asteen yhtälö
Kun kolme peräkkäistä kokonaislukua kerrotaan keskenään, ja tuloon
lisätään keskimmäinen luku, tulos on 15 kertaa keskimmäisen luvun neliö.
Mitkä luvut ovat kyseessä?
    \begin{vastaus}
	Luvut ovat $-1, 0$ ja $1$ tai $14, 15$ ja $16$.
    \end{vastaus}
\end{tehtava}

\begin{tehtava} % Tämä kai tarvitsisi polynomien jakokulman?
(K94/T2a) Polynomin $P(x)=ax^3-31x^2+1$ eräs nollakohta on $x=1$. Määritä $a$ ja ratkaise tämän jälkeen yhtälö $P(x)=0$.
\begin{vastaus}
      $a=30$ yhtälön ratkaisut ovat $1$, $\frac{1}{5}$ ja $-\frac{1}{6}$.
    \end{vastaus}
\end{tehtava}

\end{tehtavasivu}
