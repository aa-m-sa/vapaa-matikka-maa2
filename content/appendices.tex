\liitetyyli




\section{Kertaustehtäviä}


\begin{tehtavasivu}


\begin{tehtava}
   	Muistikaavat on opittava ulkoa ja niiden käytön tulee automatisoitua.
	Laske siis nämä käyttäen muistikaavoja. Tavoite on kirjoittaa vastaus suoraan ilman välivaiheita. Jos se ei vielä onnistu, yritä selvitä yhdellä välivaiheella.
    \begin{alakohdat}
        \alakohta{$(x+2)^2$}
        \alakohta{$(x-5)^2$}
        \alakohta{$(b+4)(b-4)$}
        \alakohta{$(x-1)^2$}
            \alakohta{$(2x+1)^2$}
            \alakohta{$(1-a)(1+a)$}
            \alakohta{$(x-7)^2$}
            \alakohta{$(y+1)^2$}
            \alakohta{$(x-3y)^2$}
            \alakohta{$(3x-1)^2$}
            \alakohta{$(2x+1)(2x-1)$}
            \alakohta{$(t-2)^2$}
            \alakohta{$(2x+3)^2$}
            \alakohta{$(2-a)^2$}
            \alakohta{$(5x+2)(5x-2)$}
            \alakohta{$(3-c)(3+c)$}
            \alakohta{$(10x-1)^2$}
            \alakohta{$(2a+b)^2$}            
            \alakohta{$(x+6)^2$}                                                                                                                                                        
    \end{alakohdat}
    \begin{vastaus}
        \begin{alakohdat}
            \alakohta{$x^2+4x+4$}
            \alakohta{$x^2-10x+25$}
            \alakohta{$b^2-16$}
            \alakohta{$x^2-2x+1$}
            \alakohta{$4x^2+4x+1$}
            \alakohta{$1-a^2$}
            \alakohta{$x^2-14x+49$}
            \alakohta{$y^2+2y+1$}
            \alakohta{$x^2+6xy+9y^2$}
            \alakohta{$9x^2-6x+1$}
            \alakohta{$4x^2-1$}
            \alakohta{$t^2-4t+4$}
            \alakohta{$4x^2+12x+9$}
            \alakohta{$a^2-4a+4$}
            \alakohta{$25x^2-4$}
            \alakohta{$9-c^2$}                                                                                                                                    
            \alakohta{$100x^2-20x+1$}            
            \alakohta{$4a^2+4ab+b^2$}                                                                                                                                                        
            \alakohta{$x^2+12x+36$}                                                                                                                                                        
        \end{alakohdat}
    \end{vastaus}
\end{tehtava}


\begin{tehtava}
   	Muistikaavan mukaisen lausekkeen tunnistaminen on tärkeää.
Tunnista edellisessä tehtävässä laskemasi muistikaavat ja
   	esitä lausekkeet alkuperäisessä muodossaan tulona.
    \begin{alakohdat}
            \alakohta{$1-a^2$} 
	        \alakohta{$a^2-4a+4$}
            \alakohta{$4x^2+4x+1$}
            \alakohta{$x^2-14x+49$}
            \alakohta{$100x^2-20x+1$}            
            \alakohta{$4a^2+4ab+b^2$}                                                                                                                                                        
            \alakohta{$y^2+2y+1$}
            \alakohta{$x^2+4x+4$}
            \alakohta{$4x^2-1$}
            \alakohta{$t^2-4t+4$}
            \alakohta{$25x^2-4$}
            \alakohta{$b^2-16$}
            \alakohta{$x^2-2x+1$}
            \alakohta{$9-c^2$}                                                                                                                                    
            \alakohta{$x^2+12x+36$}                                                                                                                                                        
            \alakohta{$4x^2+12x+9$}            
            \alakohta{$x^2-10x+25$}
            \alakohta{$x^2+6xy+9y^2$}
            \alakohta{$9x^2-6x+1$}
    \end{alakohdat}
    \begin{vastaus}
        \begin{alakohdat}
            \alakohta{$(1-a)(1+a)$} 
            \alakohta{$(a-2)^2$}
            \alakohta{$(2x+1)^2$}
            \alakohta{$(x-7)^2$}
            \alakohta{$(10x-1)^2$}
            \alakohta{$(2a+b)^2$}   
            \alakohta{$(y+1)^2$}
   		     \alakohta{$(x+2)^2$}
            \alakohta{$(2x+1)(2x-1)$}
             \alakohta{$(t-2)^2$}
            \alakohta{$(5x+2)(5x-2)$}
    	    \alakohta{$(b+4)(b-4)$}
 	       \alakohta{$(x-1)^2$}
            \alakohta{$(3-c)(3+c)$}
            \alakohta{$(x+6)^2$}                                                                                                                                                         
            \alakohta{$(2x+3)^2$}            
  		    \alakohta{$(x-5)^2$}  
            \alakohta{$(x-3y)^2$}
            \alakohta{$(3x-1)^2$}
         \end{alakohdat}
    \end{vastaus}
\end{tehtava}

\newpage

\begin{tehtava}
Sievennä muistikaavan avulla
    \begin{alakohdat}
            \alakohta{$(x^2-1)^2$} 
	        \alakohta{$(a^6+3b^3)^2$}
            \alakohta{$(-12-3x)(12-3x)$}
            \alakohta{$(x+\frac{1}{x})^2$}
    \end{alakohdat}
    \begin{vastaus}
        \begin{alakohdat}
            \alakohta{$x^4-2x^2+1$} 
            \alakohta{$a^{12}+6a^6b^3+9b^6$}
            \alakohta{$-(144-9x^2)=9x^2-144$}
            \alakohta{$x^2+2+\frac{1}{x^2}$}
         \end{alakohdat}
    \end{vastaus}
\end{tehtava}

\begin{tehtava} % toisen asteen yhtälö
	Yllättäviä yhteyksiä:
    \begin{alakohdat}
            \alakohta{Perustele, että $\left(2+\sqrt{3}\right)^{-1}= 2-\sqrt{3}$.} 
	        \alakohta{$\star$ Miten tämä yleistyy?}
    \end{alakohdat}

    \begin{vastaus}
    \begin{alakohdat}
            \alakohta{Tutki lukujen $2+\sqrt{3}$ ja $2-\sqrt{3}$ tuloa.} 
	        \alakohta{Yleisesti $\left(a+\sqrt{a^2-1}\right)^{-1}= a-\sqrt{a^2-1}$}
    \end{alakohdat}
    \end{vastaus}
\end{tehtava}

\begin{tehtava} %Tämä voisi olla tulon merkkisäännön kohdalla?
Osoita, että $x^2+\frac{1}{x^2}\geq 2$, kun $x \neq 0$.
    \begin{vastaus}
     Aloita tiedosta $\left(x-\frac{1}{x}\right)^2 \geq 0$ ja sievennä.
    \end{vastaus}
\end{tehtava}

\begin{tehtava} %Tämä voisi olla tulon merkkisäännön kohdalla?
Osoita, että funktio $f(x)=x^4+3x^2+1$ saa vain positiivisia arvoja.
    \begin{vastaus}
     $x^4\geq 0$ ja $x^2 \geq 0$, joten $f(x) \geq 1$.
    \end{vastaus}
\end{tehtava}

\begin{tehtava} 
$\star$ Osoita, että kun $a \geq 0$ ja $b \geq 0$, pätee \\ $\frac{a+b}{2} \geq \sqrt{ab}$. Milloin yhtäsuuruus on voimassa?
    \begin{vastaus}
     Opastus: Aloita tiedosta $\left(\sqrt{a}-\sqrt{b}\right)^2 \geq 0$ ja sievennä. Yhtäsuuruus pätee, kun $a = b$.
    \end{vastaus}
\end{tehtava}

\paragraph*{Toisen asteen yhtälö}

\begin{tehtava} % toisen asteen yhtälö
Neliön muotoisen taulun sivu on 36 cm. Taululle tehdään tasalevyinen kehys, jonka
nurkat on pyöristettu neljännesympyrän muotoisiksi. Kuinka leveä kehys on, kun sen
pinta-ala on puolet taulun pinta-alasta? (ympyrän pinta-ala on $\pi r^2$.)
    \begin{vastaus}
     $7,7$~cm
    \end{vastaus}
\end{tehtava}

\begin{tehtava} % toisen asteen yhtälö
Ratkaise yhtälö $x - 3 = \frac{1}{x}$.
    \begin{vastaus}
    $x =\frac{3 \pm \sqrt{13}}{2}$
    \end{vastaus}
\end{tehtava}

\begin{tehtava} % toisen asteen yhtälö
On olemassa viisi peräkkäistä positiivista kokonaislukua, joista kolmen
ensimmäisen neliöiden summa on yhtä suuri kuin kahden jälkimmäisen
neliöiden summa. Mitkä luvut ovat kyseessä?
    \begin{vastaus}
		$10^2+11^2+12^2 = 13^2 + 14^2$.
    	Jos negatiivisetkin luvut sallittaisiin, $(-2)^2+(-1)^2+0^2 = 1^2 + 2^2$ kävisi 			myös. Löytyykö vastaava $4 + 3$ luvun sarja? Entä pidempi?
    \end{vastaus}
\end{tehtava}

\paragraph*{Toisen asteen epäyhtälö}

\begin{tehtava} % toisen asteen epäyhtälö
Jäätelökioskin päivittäiset kiinteät kulut ovat $400$ euroa. Jokainen jäätelö maksaa
kauppiaalle $0,50$ euroa. Kun jäätelön myyntihinta on $x$ euroa, sitä myydään
$1000 - 200x$ kappaletta. 
\begin{alakohdat}
\alakohta{Millä myyntihinnoilla jäätelön myynti on kannattavaa?}
\alakohta{Millä myyntihinnalla saadaan suurin tuotto? Kuinka suuri?}
\end{alakohdat}
    \begin{vastaus}
		\begin{alakohdat}
		\alakohta{$1,00$ \euro \ $<$ myyntihinta $<$ $4,5$ \euro.}
		\alakohta{$2,75$ \euro, jolloin voitto on $612,50$ \euro. } 
		% Epäilyttävän hyvä bisnes
		\end{alakohdat}
    \end{vastaus}
\end{tehtava}


\paragraph*{Korkeamman asteen yhtälö}

\begin{tehtava} % Korkeamman asteen yhtälö
Kun kolme peräkkäistä kokonaislukua kerrotaan keskenään, ja tuloon
lisätään keskimmäinen luku, tulos on 15 kertaa keskimmäisen luvun neliö.
Mitkä luvut ovat kyseessä?
    \begin{vastaus}
	Luvut ovat $-1, 0$ ja $1$ tai $14, 15$ ja $16$.
    \end{vastaus}
\end{tehtava}

\paragraph*{Korkeamman asteen epäyhtälö}

\begin{tehtava} % Korkeamman asteen epäyhtälö
Ratkaise epäyhtälöt
		\begin{alakohdat}
		\alakohta{$x^3 + 2x^2-15x  > 0$  }
		\alakohta{$x^3-2x^2+x \leq 0$  }
		\end{alakohdat}
    \begin{vastaus}
		\begin{alakohdat}
		\alakohta{$-5<x<0$ tai $3 < x$}
		\alakohta{$x<0$ tai $x = 1$}
		\end{alakohdat}
    \end{vastaus}
\end{tehtava}



\begin{tehtava} % Korkeamman asteen epäyhtälö
Olkoon $a > 0$. Millä muuttujan $x$ arvoilla funktion
$P(x)=x^3-ax$ arvot ovat positiivisia?
    \begin{vastaus}
	$P(x)>0$ kun $x > \sqrt{a}$ tai $-\sqrt{a}<x<0$.
    \end{vastaus}
\end{tehtava}

\newpage
\section*{Kertaustehtäviä}

\subsection*{Polynomi}

\begin{tehtava} 
Laske.
		\begin{alakohdat}
		\alakohta{$x+x+x$  }
		\alakohta{$x\cdot x \cdot x$  }
		\alakohta{$-5x^2-5x^2$  }
		\alakohta{$2x\cdot 7x$  }
		\alakohta{$\frac{1}{2}x^2\cdot(-6x^3)$  }						
		\end{alakohdat}
    \begin{vastaus}
		\begin{alakohdat}
		\alakohta{$3x$}
		\alakohta{$x^3$}
		\alakohta{$-10x^2$}
		\alakohta{$14x^2$}
		\alakohta{$-3x^5$}						
		\end{alakohdat}
    \end{vastaus}
\end{tehtava}

\begin{tehtava} 
Laske.
		\begin{alakohdat}
		\alakohta{$-2x^3+4x-x^2-x+5x^3$  }
		\alakohta{$x-3-(5-x)$  }
		\alakohta{$(x^2+5x+2)-(-x^2+3x-5)$  }
		\alakohta{$3x(x-7)$  }
		\alakohta{$-2x^2(3x^5-12x^2+1)$  }						
		\end{alakohdat}
    \begin{vastaus}
		\begin{alakohdat}
		\alakohta{$3x^3-x^2+3x$}
		\alakohta{$2x-8$}
		\alakohta{$2x^2+2x+7$}
		\alakohta{$3x^2-21x$}
		\alakohta{$-6x^7+24x^4-2x^2$}						
		\end{alakohdat}
    \end{vastaus}
\end{tehtava}

\begin{tehtava} 
Kerro sulut auki.
		\begin{alakohdat}
		\alakohta{$(x-2)(x+4)$  }
		\alakohta{$(3a+b)(a-b)$  }
		\alakohta{$(a+3)^2$  }
		\alakohta{$(x-1)^2$  }
		\alakohta{$(4x+1)(4x-1)$  }						
		\end{alakohdat}
    \begin{vastaus}
		\begin{alakohdat}
		\alakohta{$x^2+2x-8$}
		\alakohta{$3a^2-2ab-b^2$}
		\alakohta{$a^2+6a+9$}
		\alakohta{$x^2-2x+1$}
		\alakohta{$16x^2-1$}						
		\end{alakohdat}
    \end{vastaus}
\end{tehtava}

\begin{tehtava} 
Jaa tekijöihin.
		\begin{alakohdat}
		\alakohta{$4x^2+x$  }
		\alakohta{$5x^3y+10xy^2$  }
		\alakohta{$y^2-9$  }
		\alakohta{$x^2-4x+4$  }
		\alakohta{$x^4-5x^3+10x-50$  }						
		\end{alakohdat}
    \begin{vastaus}
		\begin{alakohdat}
		\alakohta{$x(4x+1)$}
		\alakohta{$5xy(x^2+2y)$}
		\alakohta{$(y+3)(y-3)$}
		\alakohta{$(x-2)^2$}
		\alakohta{$(x^3+10)(x-5)$}						
		\end{alakohdat}
    \end{vastaus}
\end{tehtava}

\begin{tehtava} 
Ratkaise yhtälö \\ $(x-3)(x+2)(x-1)=0$.
    \begin{vastaus}
		$x=3$ tai $x=-2$ tai $x=1$.
    \end{vastaus}
\end{tehtava}

\begin{tehtava} 
Miksi polynomi $x^6+3x^2+5$ ei voi saada negatiivisia arvoja?
    \begin{vastaus}
		Koska $x^6\geq 0$ ja $x^2 \geq 0$. (Parilliset potenssit.)
    \end{vastaus}
\end{tehtava}

\begin{tehtava} 
		\begin{alakohdat}
		\alakohta{Osoita oikeaksi kaavat \\
$a^3-b^3=(a-b)(a^2+ab+b^2)$ ja \\
$a^3+b^3=(a+b)(a^2-ab+b^2)$}
		\alakohta{Jaa $x^6-y^6$ neljään tekijään.}						
		\end{alakohdat}
    \begin{vastaus}
		\begin{alakohdat}
		\alakohta{Opastus: Kerro sulut auki.}
		\alakohta{$(x-y)(x^2+xy+y^2)(x+y)(x^2-xy+y^2)$}						
		\end{alakohdat}
    \end{vastaus}
\end{tehtava}

\begin{tehtava} 
Kuvassa on polynomin $P(x)$ kuvaaja.
\begin{kuvaajapohja}{0.8}{-3}{3}{-2}{4}
				\kuvaaja{x*(x-1)*(x+2)}{}{black}
\end{kuvaajapohja}

		\begin{alakohdat}
		\alakohta{Mitkä ovat polynomin nollakohdat?}
		\alakohta{Millä muuttujan $x$ arvoilla $P(x)>0$?}
		\alakohta{Kuinka monta ratkaisua yhtälöllä $P(x)=1$ on?}
		\end{alakohdat}
    \begin{vastaus}
		\begin{alakohdat}
		\alakohta{$x=-2$, $x=0$ ja $x=1$}
		\alakohta{$-2<x<0$ tai $1 < x$}
		\alakohta{Vähintään 3. (Kuvaajassa näkyvän alueen 
		ulkopuolella voisi olla lisää.)}
		\end{alakohdat}
    \end{vastaus}
\end{tehtava}

\begin{tehtava} %muistikaavat
Jaa tekijöihin \\ $(3x^2-7y^2+5)^2-(x^2-9y^2-5)^2$.
    \begin{vastaus}
		$8(x-2y)(x+2y)(x^2+y^2+7)$. \\
    Opastus: Älä kerro aluksi sulkuja auki vaan käytä heti muistikaavaa.
    \end{vastaus}
\end{tehtava}

\subsection*{Ensimmäinen aste}

\begin{tehtava} 
Ratkaise yhtälöt.
		\begin{alakohdat}
		\alakohta{$3x-5(x-2)=3-(-x)$ }
		\alakohta{$x(x-6) = x^2+5$  }
		\alakohta{$ \frac{2x}{3}-\frac{x-3}{4}=7$ }
		\end{alakohdat}
    \begin{vastaus}
		\begin{alakohdat}
		\alakohta{$x = \frac{7}{3}$}
		\alakohta{$x=-\frac{5}{6}$}
		\alakohta{$15$}			
		\end{alakohdat}
    \end{vastaus}
\end{tehtava}

\begin{tehtava} 
Mitä tarkoittaa
		\begin{alakohdat}
		\alakohta{$x \in [2,7]$ }
		\alakohta{$y \in ]-3,0]$  }
		\alakohta{$z \in ]-\infty, 5[$ ?}
		\end{alakohdat}
    \begin{vastaus}
		\begin{alakohdat}
		\alakohta{$2 \leq x \leq 7$}
		\alakohta{$-3 < y \leq 0$}
		\alakohta{$z < 5$}			
		\end{alakohdat}
    \end{vastaus}
\end{tehtava}

\begin{tehtava} 
Ratkaise epäyhtälöt.
		\begin{alakohdat}
		\alakohta{$-3x < 6$ }		
		\alakohta{$5x-2 > 7x+3$ }
		\alakohta{$ -2(x+3)  \leq x-(5-x)$  }
		\end{alakohdat}
    \begin{vastaus}
		\begin{alakohdat}
		\alakohta{$ x > -2$}	
		\alakohta{$x < -\frac{5}{2}=-2\frac{1}{2}$}
		\alakohta{$x \geq -\frac{1}{4}$}
		\end{alakohdat}
    \end{vastaus}
\end{tehtava}

\begin{tehtava} 
Polkupyörävuokraamo A laskuttaa pyörän vuokrasta $5$~\euro \ ja $2,5$~\euro \
jokaisesta täydestä tunnista. Vuokraamo B laskuttaa $11,50$~\euro \ ja $1,5$~\euro \ jokaisesta
täydestä tunnista. Kuinka monen tunnin vuokrassa vuokraamo B on edullisempi?
    \begin{vastaus}
	7 h tai sitä pidemmissä vuokrissa.
    \end{vastaus}
\end{tehtava}

\begin{tehtava} 
Ratkaise epäyhtälö \\
$ax \geq a-2x$ \\ parametrin $a$ kaikilla arvoilla.
    \begin{vastaus}
        $x \geq \frac{a}{a+2}$, kun $a > -2$ \\
        $x \leq \frac{a}{a+2}$, kun $a < -2$ \\
    $x \in \mathbb{R}$, kun $a = -2$ \\
	\end{vastaus}
\end{tehtava}

\subsection*{Toinen aste}

\begin{tehtava} 
Ratkaise yhtälöt.
		\begin{alakohdat}
		\alakohta{$x^2-13=0$ }
		\alakohta{$5x^+2x=0$  }
		\alakohta{$2x^2+5x-3=0$}
		\end{alakohdat}
    \begin{vastaus}
		\begin{alakohdat}
		\alakohta{$x= \pm \sqrt{13}$}
		\alakohta{$x=0$ tai $x=-\frac{2}{5}$}
		\alakohta{$x=-3$ tai $x= \frac{1}{2}$}			
		\end{alakohdat}
    \end{vastaus}
\end{tehtava}

\begin{tehtava} 
Ratkaise epäyhtälöt.
		\begin{alakohdat}
		\alakohta{$x^2-x-6<0$ }
		\alakohta{$x^2 \geq 5x$  }
		\end{alakohdat}
    \begin{vastaus}
		\begin{alakohdat}
		\alakohta{$ -2 < x < 3 $}
		\alakohta{$x \leq 0$ tai $x \geq 5$}
	\end{alakohdat}
    \end{vastaus}
\end{tehtava}

\begin{tehtava} 
Maijan ja Veeran ikien summa on 30. Ikien tulo on yli 186. Minkä ikäisiä tytöt voivat olla?
    \begin{vastaus}
	Vähintään 9, korkeintaan 21. Ratkeaa epäyhtälöstä $x(30-x)>186$.
	\end{vastaus}
\end{tehtava}

\begin{tehtava} 
Millä vakion $k$ arvolla yhtälöllä \\ $kx^2+kx=x-3$ on tasan yksi ratkaisu?
    \begin{vastaus}
		$k = 7 \pm 4 \sqrt{3}$. Diskriminantti on $D = (k-1)^2-4\cdot k \cdot 3$.
    \end{vastaus}
\end{tehtava}

\begin{tehtava} 
Polynomilla $P(x)=x^2-3x+c$ on tekijä $x+5$. Mikä on $c$?
    \begin{vastaus}
		$c=-40$
    \end{vastaus}
\end{tehtava}

\begin{tehtava} 
Suorakulmion $A$ sivut ovat $9$ ja $x^2+1$, suorakulmion $B$ sivut $5x+5$
ja $5-x$. Millä luvun $x$ arvoilla suorakulmion $A$ ala on suurempi?
    \begin{vastaus}
	$-1 < x < -\frac{4}{7}$ tai $2 < x < 5$. Huomioi, että suorakulmion $B$
    sivut ovat positiiviset vain, kun $-1<x<5$.
    \end{vastaus}
\end{tehtava}

\subsection*{Korkeampi aste}

\begin{tehtava} 
Ratkaise yhtälö $2x^7=5x^2$.
    \begin{vastaus}
		$x=0$ tai $x=\sqrt[5]{\frac{5}{2}}$
    \end{vastaus}
\end{tehtava}

\begin{tehtava} 
Anna esimerkki (jos mahdollista)
		\begin{alakohdat}
		\alakohta{4. asteen yhtälöstä, jolla ei ole ratkaisua }
		\alakohta{5. asteen yhtälöstä, jolla on tasan kaksi ratkaisua}
		\alakohta{3. asteen yhtälöstä, jolla on tasan neljä ratkaisua}
		\end{alakohdat}
    \begin{vastaus}
		\begin{alakohdat}
		\alakohta{esimerkiksi $x^4=-1$ }
		\alakohta{esimerkiksi $x^4(x+1)=0$ }
		\alakohta{mahdotonta}		
		\end{alakohdat}
    \end{vastaus}
\end{tehtava}

\begin{tehtava} 
Ratkaise yhtälö $x^4=x^2+6$.
    \begin{vastaus}
		$x=\pm \sqrt{3}$
    \end{vastaus}
\end{tehtava}

\begin{tehtava} 
Ratkaise yhtälö
$x^7=5x^5-x^6$.
     \begin{vastaus}
		$x=0$ tai $x\frac{-1 \pm \sqrt{21}}{2}$ 
    \end{vastaus}
\end{tehtava}

\begin{tehtava} % Korkeamman asteen epäyhtälö
Mitkä luvut ovat kuutiotaan suurempia?
    \begin{vastaus}
	Luvut, jotka ovat pienempiä kuin $-1$ ja luvut välillä $]0,1[$.
    \end{vastaus}
\end{tehtava}

\begin{tehtava} 
Ratkaise epäyhtälöt
		\begin{alakohdat}
		\alakohta{$x^4 < 5x $}
		\alakohta{$2x^3 \leq 3x-5x^2$}
		\end{alakohdat}
     \begin{vastaus}
		\begin{alakohdat}
		\alakohta{$0<x<\sqrt[3]{5}$}
		\alakohta{$x<-3$ tai $0<x\frac{1}{2}$}
		\end{alakohdat}
    \end{vastaus}
\end{tehtava}

\begin{tehtava} 
Kuution tilavuus (kuutiometreinä) on sama kuin sen särmien pituuksien summa (metreinä). Kuinka pitkä on kuution särmä?
    \begin{vastaus}
		$\sqrt{12}$~m $\approx 3,46$~m
    \end{vastaus}
\end{tehtava}

\subsection*{Sekalaisia}

\begin{tehtava} 
Sievennä:
		\begin{alakohdat}
			\alakohta{$(a^2-1)^2+(a^2+1)^2-2(a^4+1)$}
			\alakohta{$(x+y)^2-4xy$}
		\end{alakohdat}
	\begin{vastaus}
		\begin{alakohdat}
			\alakohta{$0$}
			\alakohta{$(x-y)^2$}
		\end{alakohdat}
    \end{vastaus}
\end{tehtava}

\begin{tehtava} 
Ratkaise yhtälö
$2x^5-(x+6)^5=0$.
    \begin{vastaus}
	$x=\frac{6}{\sqrt[5]{2}-1}$
    \end{vastaus}
\end{tehtava}

\begin{tehtava} 
Suorakulmion piiri on 34 ja lävistäjä 13. Ratkaise suorakulmion sivut.
    \begin{vastaus}
	Sivut ovat $5$ ja $12$.
    \end{vastaus}
\end{tehtava}

\begin{tehtava} % Korkeamman asteen yhtälö
Etsi yhtälön $x^5-5x^4+6x^3-1=0$ kaikkien kolmen ratkaisun likiarvot
laskimella tai tietokoneen avulla. Anna vastaukset
kahden desimaalin tarkkuudella.
    \begin{vastaus}
	$x \approx 0,69$, $x \approx 1,86$ tai $x \approx 3,03$.
    \end{vastaus}
\end{tehtava}


\end{tehtavasivu}



\section{Tehtäviä ylioppilaskokeista}

\subsubsection*{Lyhyen oppimäärän tehtäviä}

k = lyhyt matematiikka
K = pitkä matematiikka

% \begin{tehtava}
% (K2011/1a) \\ Ratkaise yhtälö $4x+(5x-4) = 12+3x$.
% \begin{vastaus}
% $x=\frac{16}{3} = 5\frac13$
% \end{vastaus}
% \end{tehtava}

\begin{tehtava}
(k2011/1b) Sievennä lauseke $x^2+x-(x^2-x)$.
\begin{vastaus}
$2x$
\end{vastaus}
\end{tehtava}

\begin{tehtava}
(k2011/2b) \\ Sievennä lauseke $(\sqrt{x}-1)^2+2\sqrt{x}$.
\begin{vastaus}
$x+1$
\end{vastaus}
\end{tehtava}

\begin{tehtava}
(s2011/1c) \\ Ratkaise yhtälö $x^2-3(x+3) = 3x-18$.
\begin{vastaus}
$x=3$
\end{vastaus}
\end{tehtava}

\begin{tehtava}
(k2012/1a) Ratkaise yhtälö $7x+3 = 31$.
\begin{vastaus}
$x = 4$
\end{vastaus}
\end{tehtava}

\begin{tehtava}
(s2012/1a) Ratkaise yhtälö $x^2-2x = 0$.
\begin{vastaus}
$x=0$ tai $x=2$
\end{vastaus}
\end{tehtava}

\begin{tehtava}
(k2013/2a) Millä muuttujan $x$ arvoilla $4x+17$ on suurempi kuin $2-x$?
\begin{vastaus}
$x>-3$
\end{vastaus}
\end{tehtava}

\begin{tehtava}
(k2013/2b) Ratkaise yhtälö $x^2+14x=-49$.
\begin{vastaus}
$x=-7$
\end{vastaus}
\end{tehtava}

\begin{tehtava}
(s2013/1a) Ratkaise yhtälö $(x-2)^2=4$.
\begin{vastaus}
$x=0$ tai $x=4$ 
\end{vastaus}
\end{tehtava}

\subsubsection*{Pitkän oppimäärän tehtäviä}

\begin{tehtava}
(K1988/3) Millä $a$:n arvoilla yhtälön $x^2+(3a+1)x+81=0$ juuret ovat reaaliset?
\begin{vastaus}
$a \leq -\frac{19}{3}$ tai $a \geq \frac{17}{3}$
\end{vastaus}
\end{tehtava}

\begin{tehtava}
(K2007/1a) Ratkaise yhtälö $7x^2-6x=0$.
\begin{vastaus}
$x=0$ tai $x=\frac{6}{7}$
\end{vastaus}
\end{tehtava}

\begin{tehtava}
(S2007/1a) Ratkaise epäyhtälö $2-3x>4x$.
\begin{vastaus}
$x< \frac{2}{7} $
\end{vastaus}
\end{tehtava}

\begin{tehtava}
(K2008/1a) Ratkaise yhtälö $2x^2=x+1$.
\begin{vastaus}
$x=1$ tai $x=-\frac{1}{2}$
\end{vastaus}
\end{tehtava}

\begin{tehtava}
(S2008/1a) Ratkaise epäyhtälö $\frac{1}{2} - \frac{x}{3} > \frac{3}{4}$.
\begin{vastaus}
$x<-\frac{3}{4}$
\end{vastaus}
\end{tehtava}

% Erillisten murtolausekkeiden laventamista samannimisiksi -> ehkä enemmän MAA1-asiaa.
% \begin{tehtava}
% (S2008/1b) Sievennä lauseke $\frac{1}{x}-\frac{1}{x^2}+ \frac{1+x}{x^2}$.
% \begin{vastaus}
% $ \frac{2}{x}$
% \end{vastaus}
% \end{tehtava}

\begin{tehtava}
(S2008/3b) Ratkaise yhtälö $4x^3-5x^2=2x-3x^3$.
\begin{vastaus}
$x=-\frac{2}{7}$ tai $x=0$ tai $x=1$
\end{vastaus}
\end{tehtava}

\begin{tehtava}
(K2009/1b) Ratkaise epäyhtälö$(x-3)^2>(x-1)(x+1)$.
\begin{vastaus}
$x<\frac{5}{3}$
\end{vastaus}
\end{tehtava}

\begin{tehtava}
(S2009/1a) Ratkaise yhtälö $(x-2)(x-3)=6$. 
\begin{vastaus}
$x=0$ tai $x=5$
\end{vastaus}
\end{tehtava}

\begin{tehtava}
(S2009/1b) Ratkaise yhtälö $\frac{x}{x-3}-\frac{1}{x}=1$.
\begin{vastaus}
$x=-\frac{3}{2}$
\end{vastaus}
\end{tehtava}

\begin{tehtava}
(S2009/2a) Ratkaise epäyhtälö $6(x-1)+4 \geq 3(7x+1)$. 
\begin{vastaus}
$x \leq -\frac{1}{3}$
\end{vastaus}
\end{tehtava}

% 7. kurssin asiaa:
% \begin{tehtava}
% (S2009/8) Ratkaise epäyhtälö $\frac{-x^2+x+2}{x^3+2x^2-3x}>0$.
% \begin{vastaus}
% $x<-3$ tai $-1<x<0$ tai $1<x<2$
% \end{vastaus}
% \end{tehtava}

\begin{tehtava}
(K2010/1a) Ratkaise yhtälö $7x^7+6x^6=0$.
\begin{vastaus}
$x=0$ tai $x=-\frac{6}{7}$
\end{vastaus}
\end{tehtava}

\begin{tehtava}
(K2010/1b) Sievennä lauseke $(\sqrt{a}+1)^2-a-1$.
\begin{vastaus}
$2\sqrt{a}$
\end{vastaus}
\end{tehtava}

\begin{tehtava}
(K2010/1c) Millä $x$:n arvoilla pätee $\frac{3}{3-2x}<0$?
\begin{vastaus}
$x>\frac{3}{2}$
\end{vastaus}
\end{tehtava}

\begin{tehtava}
(K2010/3b) Määritä toisen asteen yhtälön $x^2+px+q=0$ kertoimet $p$ ja $q$, kun yhtälön juuret ovat $-2-\sqrt{6}$ ja $-2+\sqrt{6}$.
\begin{vastaus}
$p=4$, $q=-2$
\end{vastaus}
\end{tehtava}

\begin{tehtava}
(S2010/1a) Sievennä lauseke $(a+b)^2-(a-b)^2$.
\begin{vastaus}
$4ab$
\end{vastaus}
\end{tehtava}

\begin{tehtava}
(S2010/2a) Ratkaise epäyhtälö $x\sqrt{7}-3 \leq 4x$.
\begin{vastaus}
$x \geq \frac{3}{\sqrt{7}-4}$
\end{vastaus}
\end{tehtava}

\begin{tehtava}
(S2010/2c) Ratkaise yhtälö $x^4-3x^2-4=0$.
\begin{vastaus}
$x=2$ tai $x=-2$
\end{vastaus}
\end{tehtava}

% Murtolukujen jakokulmaa ei tule tässä kurssissa.
% \begin{tehtava}
% (S2010/12) Määritä $a$ siten, että polynomi $P(x)=2x^4-3x^3-7x^2+a$ on jaollinen binomilla $2x-1$. Määritä tätä $a$:n arvoa vastaavat yhtälön $P(x)=0$ juuret.
% \begin{vastaus}
% $a=2$. Yhtälön $P(x)=0$ juuret ovat $x=\frac{1}{2}$, $x=-1$, $x=1+\sqrt{3}$ ja $x=1-\sqrt{3}$.
% \end{vastaus}
% \end{tehtava}

\begin{tehtava}
(K2011/1b) Ratkaise epäyhtälö $x^2-2 \leq x$.
\begin{vastaus}
$-1 \leq x \leq 2$
\end{vastaus}
\end{tehtava}

\begin{tehtava}
  (S2011/3b) Ratkaise epäyhtälö $\frac{2x+1}{x-1} \geq 3$.
\begin{vastaus}
$1<x \leq 4$
\end{vastaus}
\end{tehtava}

\begin{tehtava}
(S2012/1a) \\ Ratkaise yhtälö $2(1-3x+3x^2) = 3(1+2x+2x^2)$.
\begin{vastaus}
$x=-\frac{1}{12}$
\end{vastaus}
\end{tehtava}

\begin{tehtava}
(S2013/1a) \\ Ratkaise yhtälö $x^2+6x=2x^2+9$.
\begin{vastaus}
$x=3$
\end{vastaus}
\end{tehtava}

\begin{tehtava}
(S2013/1b) \\ Ratkaise yhtälö $\frac{1+x}{1-x}=\frac{1-x^2}{1+x^2}$.
\begin{vastaus}
$x=0$ tai $x=-1$
\end{vastaus}
\end{tehtava}

\begin{tehtava}
(S2013/1c) \\ Esitä polynomi $x^2-9x+14$ ensimmäisen asteen polynomien tulona.
\begin{vastaus}
$(x-2)(x-7)$
\end{vastaus}
\end{tehtava}

\begin{tehtava}
(K2013/1b) \\ Ratkaise epäyhtälö $\frac{3}{5}x-\frac{7}{10} < -\frac{2}{15}x$.
\begin{vastaus}
$x<\frac{21}{22}$
\end{vastaus}
\end{tehtava}

\begin{tehtava}
(K2013/3a) \\ Laske lausekkeen $(\sqrt{a}+\sqrt{b})^2$ tarkka arvo, kun positiiviset luvut $a$ ja $b$ ovat toistensa käänteislukuja ja lukujen $a$ ja $b$ keskiarvo on $2$.
\begin{vastaus}
$6$
\end{vastaus}
\end{tehtava}

\begin{tehtava}
(K2013/*14a) \\ Jaa $P(x)=x^2+x-2$ ensimmäisen asteen tekijöihin. (2 p.)
\begin{vastaus}
$(x+2)(x-1)$
\end{vastaus}
\end{tehtava}

\begin{tehtava}
(K2013/*14b) \\ Olkoon $P(x)=x^2+x-2$. Määritä sellaiset vakiot $A$ ja $B$, että $\frac{1}{P(x)}=\frac{A}{x-1}+\frac{B}{x+2} $ kaikilla $x \geq 2$. (2 p.)
\begin{vastaus}
$A= \frac{1}{3}$ ja $B=- \frac{1}{3}$
\end{vastaus}
\end{tehtava}


% (S2011, 1b) Suorakulmaisen kolmion hypotenuusan pituus on 5 % Kurssi 3
%   ja toisen kateetin pituus 2. Laske toisen kateetin pituus.

  

%fixme: etsi lisää yo-tehtäviä, näitä on kyllä olemassa

\section{Tavoittele valaistumista}


Tässä on joitakin tehtäviä, jotka on arvioitu hauskoiksi ja hyödyllisiksi kaikkein osaavimmille opiskelijoille. Niiden parissa vietetty aika ei mene hukkaan, vaikkei tehtävä ratkeaisikaan.

\begin{tehtavasivu}

\begin{tehtava}
    \begin{enumerate}[a)]
        \item Osoita, että jos polynomi $P$ jaetaan polynomilla $x-1$, niin jakojäännös on polynomin $P$ kertoimien summa.
        \item Päättele tästä, että kokonaisluku $n$ on jaollinen yhdeksällä vain jos sen kymmenjärjestelmäesityksen numeroiden summa on jaollinen yhdeksällä.
    \end{enumerate}
\end{tehtava}

\begin{tehtava} %Ehkä käsitteellisesti vaikea
    $P$ on toisen asteen polynomi, jonka vakiotermi on $1$. Polynomi $Q$ määritellään lausekkeella $Q(x)=P(x+1)-P(x)$ ja siitä tiedetään, että $Q(0)=7$ ja $Q(1)=13$. Määritä polynomin $P$ lauseke.
    \begin{vastaus}
        $P(x) = 3x^2+4x+1$
    \end{vastaus}
\end{tehtava}

\begin{tehtava} %Vaikea!
    Etsi kaikki positiiviset kokonaisluvut $x$ ja $y$, joille pätee $9x^2-y^2=17$.
    \begin{vastaus}
    Opastus: jaa yhtälön vasen puoli tekijöihin muistikaavalla. 
    Ainoa ratkaisu on $x = 3$, $y=8$.
    \end{vastaus}
\end{tehtava}

\begin{tehtava} % Sikavaikea (tällä tasolla)
%täydentyy kahdeksi neliöksi, joiden summa on 0
    Ratkaise $x$ ja $y$ yhtälöstä $y^2+2xy+x^4-3x^2+4=0$.
    \begin{vastaus}
        $x=\sqrt{2}, y=-\sqrt{2}$ tai $x=-\sqrt{2}, y=\sqrt{2}$
    \end{vastaus}
\end{tehtava}

\begin{tehtava} % Kaunis
    Ratkaise $x$ ja $y$ yhtälöstä $2x^4+2y^4=4xy-1$. %lisää tai vähennä kiva termi puolittain
    \begin{vastaus}
        $x=\frac{\sqrt{2}}{2}, y=\frac{\sqrt{2}}{2}$ tai $x=-\frac{\sqrt{2}}{2}, y=-\frac{\sqrt{2}}{2}$
    \end{vastaus}
\end{tehtava}

\begin{tehtava} %Erittäin tärkeä ja hyödyllinen, mutta liekö vielä paikallaan... esitystä voisi ehkä muuttaa
Toisen asteen polynomi $P_1 = (ax-b)^2$ ($a \neq 0 $) on binomin neliönä aina epänegatiivinen. Tulon nollasäännön avulla nähdään, että sillä on tasan yksi nollakohta $x = \frac{b}{a}$. Tällöin polynomin diskriminantin arvo on 0, mikä voidaan nähdä myös suoraan laskemalla.

Polynomi $P_2 = (a_1x-b_1)^2+(a_2x-b_2)^2$ ($a_1,a_2 \neq $) on niin ikään aina epänegatiivinen, mutta sillä ei välttämättä ole nollakohtaa; se on epänegatiivisten termien summa, joka voi olla nolla vain, jos kaikkien termien neliöt ovat nollia. Koska jokainen binomin neliö voi saavuttaa nollan vain yhdessä pisteessä, ne voivat kaikki saavuttaa nollan korkeintaan yhdessä pisteessä. Mutta tällöin polynomin diskriminantin on oltava korkeintaan nolla. Muodosta polynomin $P_2$ diskriminantti, mikä epäyhtälö seuraa?

Saman päättelyn voi yleistää myös $n$:n binomin neliöiden summasta muodostetulle polynomille $P_n = (a_1x-b_1)^2+(a_2x-b_2)^2+\ldots+(a_nx-b_n)^2$. Vastaavalla päättelyllä polynomilla on korkeintaan yksi nollakohta, joten sen diskriminantti on epäpositiivinen. Laske $P_n$:n diskriminantti. Olet todistanut kuuluisan Cauchy-Schwarzin epäyhtälön:
\[
(a_1^2+a_2^2+\ldots+a_n^2)(b_1^2+b_2^2+\ldots+b_n^2) \geq (a_1b_1+a_2b_2+\ldots+a_nb_n)^2
\]

\end{tehtava}

\end{tehtavasivu}





% Lisätään, kun niitä on valmiina, nyt ei yhtään kokonaista
%\section{Harjoituskokeita}

%\subsection*{Harjoituskoe 1}

\begin{enumerate}
\item Ratkaise epäyhtälöt.\\ a) $1-\dfrac{1-x}{6}<x$\\ b) $(x+1)(x^2-2x-1)\geq0$
\item Ratkaise yhtälöt.\\ a) $4x^2-1=0$\\ b) $x^3=-3x$\\ c) $2y^2=y-8$
\item Millä parametrin $k$:n arvoilla yhtälöllä $kx^2-(k+1)x+1=0$ on kaksi erisuurta reaalijuurta? 
\item Muodosta funktio, joka esittää...

\end{enumerate}

\subsection*{Harjoituskoe 2}

\begin{enumerate}
\item Ratkaise yhtälöt.\\ a) $x^2-5x=0$\\ b) $x^4-1=0$\\ c) $(x-1)(x+4) = x(x-5)$
\item Ratkaise epäyhtälöt.\\ a) $x^2-8\geq0$\\ b) $x^2-8\geq(x-3)^2$\\ c) $x^2-6x+9\leq0$
\item Millä vakion $t$ arvoilla yhtälöllä $tx^2+tx-6=0$ ei ole ratkaisuja?
\item Muodosta funktio, joka esittää...
\end{enumerate}


\subsection*{Harjoituskoe 3}

\begin{enumerate}
\item Jaa polynomi $t^5-t^4+t^3-t^2$tekijöihin. Kuinka monta reaalista nollakohtaa polynomilla on?
\item Olkoon $a > 0$. Millä muuttujan $x$ arvoilla funktion $Q(x)=x3−ax^2$ arvot ovat epänegatiivisia?
\item Millä vakion $r$ arvoilla yhtälöllä $rx^2+rx-1=0$ ei ole reaaliratkaisuja?
\item Muodosta funktio, joka esittää...
\end{enumerate}


\subsection*{Harjoituskoe 4}

\begin{enumerate}


\item Ratkaise epäyhtälöt. \\
a) $2x^3 \geq x$ \\
b) $y^2 \leq 3y -9 $

\item Millä vakion $a$ arvoilla yhtälöllä $x^2+ax+a=0$ on kaksoisjuuri?
\item Muodosta funktio, joka esittää...

\item Johda toisen asteen yhtälön ratkaisukaava lähtien yhtälön normaalimuodosta $ax^2+bx+c=0$, missä $a, b$ ja $c$ ovat reaalisia vakioita ja $a \neq 0$.
\end{enumerate}


\section{Lisämateriaalia}

%poistin virheellisenä ja tarpeettomana T: Jokke
%\input{appendices/tulonolla_todistus}

\subsection*{Toisen asteen polynomin kuvaaja}
\label{paraabeli_tod}

Tässä liitteessä perustellaan, miksi kaikkien toisen asteen polynomifunktioiden kuvaajat näyttävät samalta. Lisäksi tarkastellaan, mitkä tekijät vaikuttavat kuvaajien muotoon.

\underline{Funktio $P(x)=x^2$}

Aloitetaan funktiosta $P(x)=x^2$. Mitä tiedämme siitä piirtämättä kuvaajaa?
\begin{itemize}
\item Algebrallisen päättelyn avulla voimme todeta, että funktion pienin arvo on $P(0) = 0$, sillä tulon merkkisäännön perusteella lauseke $x^2$ ei voi saada negatiivisia arvoja – ei ole olemassa sellaista reaalilukua $x$, jonka neliö olisi negatiivinen. (Kompleksiluvuilla tällaista rajoitusta ei ole, tästä lisää liitteessä.) Kuvaajan alin kohta sijoittuu siis origoon ja kuvaajan kaikki muut pisteet sen yläpuolelle.
\item Kun $x > 0$, lauseke $x^2$  on sitä
suurempi, mitä suurempi $x$ on. Tästä tiedämme, että nollasta oikealle siirryttäessä funktion kuvaaja nousee.
\item Koska $(-x)^2 = x^2$, kuvaaja on symmetrinen $y$-akselin suhteen.
\end{itemize}

Näiden tietojen avulla voimme päätellä, että funktion kuvaaja koostuu kahdesta
identtisestä haarasta, jotka kohtaavat, kun $x=0$. Mitä kauempana $x$ on nollasta, sitä suurempia ovat funktion arvot. Tämän kaiken voi päätellä jo ennen kuvaajan piirtämistä.

Merkitsemällä koordinaatistoon yhtälön $y=x^2$ toteuttavia pisteitä, muodostuu lopulta funktion kuvaaja:
\begin{center}
\begin{kuvaajapohja}{2}{-2}{2}{-1}{4}
  \kuvaaja{x**2}{$f(x)=x^2$}{blue}
\end{kuvaajapohja}
\end{center}

% FIXME: ongelmallista: ''alaspäin aukeava'' vasta seuraavaksi
Kuvaajan muoto on geometriselta nimeltään \emph{paraabeli}. Paraabeleja esiintyy monessa yhteydessä: esimerksi polttopeilin ja radioteleskoopin pinta kaareutuu paraabelin muotoisena. Samoin ilmaan heitetyn kappaleen rata on likimain alaspäin aukeava paraabeli, kun ilmanvastus on pieni.

\underline{Funktio $P(x)=ax^2, \quad a\neq 0$}

Polynomien $P(x)=ax^2$ arvot ovat ovat lausekkeeseen $x^2$ nähden $a$-kertaisia. Muuten paraabelin symmetrisyys ja muut keskeiset ominaisuudet säilyvät.

\begin{itemize}
\item Kun $a > 0$, myös $ax^2\geq 0$, joten pienin arvo on yhä $0$.\\
Paraabeli on \termi{ylöspäin aukeava}{ylöspäin aukeava}.
\item Kun $a < 0$, tulon merkkisäännön nojalla $ax^2 \leq 0$. \\
 Nyt $P(0)=0$ onkin suurin arvo, ja funktion arvot ovat sitä pienempiä,
mitä enemmän $x$ poikkeaa nollasta. \\
Paraabeli on \termi{alaspäin aukeava}{alaspäin aukeava}.
\item Mitä enemmän $a$ poikkeaa nollasta, sitä nopeammin funktion arvot
muuttuvat $x$:n muuttuessa ja sitä kapeampi paraabeli on.
\end{itemize}

\begin{center}
$a>0$, paraabeli aukeaa ylöspäin:\\
\begin{tabular}{cc}
$f(x)=\frac{1}{2}x^2$& $f(x)=2x^2$ \\
\begin{kuvaajapohja}{1}{-2}{2}{-1}{4}
  \kuvaaja{0.5*x**2}{}{blue}
\end{kuvaajapohja} &
\begin{kuvaajapohja}{1}{-2}{2}{-1}{4}
  \kuvaaja{2*x**2}{}{blue}
\end{kuvaajapohja}
\end{tabular}

$a<0$, paraabeli aukeaa alaspäin:\\
\begin{tabular}{cc}
$f(x)=-\frac{1}{2}x^2$ & $f(x)=-2x^2$ \\
\begin{kuvaajapohja}{1}{-2}{2}{-4}{1}
  \kuvaaja{-0.5*x**2}{}{blue}
\end{kuvaajapohja} &
\begin{kuvaajapohja}{1}{-2}{2}{-4}{1}
  \kuvaaja{-2*x**2}{}{blue}
\end{kuvaajapohja}
\end{tabular}
\end{center}

\underline{Funktio $P(x)=ax^2+c$}

Lisäämällä vakiotermi $c$ saadaan $P(x)=ax^2+c$. Vakion lisääminen nostaa tai laskee funktion kuvaajaa (riippuen siitä, onko $c > 0$ tai $c<0$), joten kuvaajan muoto ei muutu.

\underline{Funktio $P(x)=ax^2+bx+c$}

Miksi sitten täydellisen toisen asteen polynomin $P(x)=ax^2+bx+c$ kuvaaja on myös paraabeli? Muokataan lauseketta, aloitetaan ottamalla yhteinen tekijä:
\begin{align*}
P(x) &=ax^2+bx+c \\
&= a\left(x^2 +\frac{b}{a}x\right) + c  \quad &
\text{ lavennetaan } \frac{b}{a} \text{ kahdella} \\
&= a\left(x^2 +2\cdot x \cdot \frac{b}{2a} \quad\right) + c  &
\text{ täydennetään neliöksi lisäämällä} \left( \frac{b}{2a} \right)^2 \\
&= a \Bigg( \underbrace{x^2 +2\cdot x \cdot \frac{b}{2a}+\left(\frac{b}{2a} \right)^2}_{\left( x+\frac{b}{2a} \right)^2}
- \left(\frac{b}{2a}\right)^2 \Bigg)  + c \\
&= a \left( \left( x + \frac{b}{2a} \right )^2-\frac{b^2}{4a^2} \right) + c &
\text{ kerrotaan sulut auki } \\
&= a \underbrace{\left(  x + \frac{b}{2a} \right)^2}_{\text{neliö}}-
\underbrace{\frac{b^2}{4a} + c}_{\text{vakio}}
\end{align*}

Tästä neliöksi täydennetyksi muodosta nähdään, että $P(x)$ on muotoa
$a\cdot$neliö + vakio. Kuvaaja on siis samanlainen kuin tapauksessa
$ax^2+c$, huippu on vain siirtynyt.
Koska neliö $\geq 0$, saadaan

\begin{itemize}
\item Kun $a>0$, kyseinen vakio on polynomin pienin arvo ja kuvaaja on
ylöspäin aukeava paraabeli.
\item Kun $a<0$, kyseinen vakio on suurin arvo ja kuvaaja on alaspäin
aukeava paraabeli.
\end{itemize}

Paraabelin \termi{huippu}{huippu} (eli kuvaajan piste, jossa suurin tai pienin arvo
saavutetaan) on aina kohdassa
$x=-\frac{b}{2a}$, koska silloin neliö on nolla.

%Toisen asteen polynomifunktio on muotoa $f(x)=ax^2+bx+c$, jossa $a,b,c \in \R$ ja $a \neq 0$. Toisen asteen polynomifunktion kuvaaja on paraabeli. Toisen asteen polynomifunktioita käytetään matemaattisessa mallinnuksessa talouden, tieteen ja tekniikan aloilla. Esimerkiksi heittoliikkeessä olevan kappaleen lentorata on aina paraabelin muotoinen. \\
%\textbf{Esimerkki 1.}
%a) Piirrä funktion $f(x)=x^2-2$ kuvaaja. \\
%b) Ratkaise funktion $f$ nollakohdat. \\ \\
%
%\begin{kuvaajapohja}{1}{-2}{2}{-3}{1}
%  \kuvaaja{x**2-2}{$f(x)=x^2-2$}{blue}
%\end{kuvaajapohja}
%
%Funtkion kuvaaja on ylöspäin aukeava paraabeli, joka leikkaa x-akselin kohdissa joissa $f(x)=0$. \\
%Graafisesti funktion nollakohdat saadaan ratkaistua kuvaajasta. Kuvaajasta nähdään, että funktion nollakohdat ovat $x_1 \approx 1,4$ ja $x_2 \approx -1,4$. \\
%Algebrallisesti saadaan ratkaistua, että funktion nollakohdat ovat
%\begin{align*}
%f(x)&=0 \\
%x^2-2&=0 \\
%x^2&=2 \\
%x&= \pm \sqrt[]{2}
%\end{align*}
%Funktion $f$ kuvaajasta huomataan, että se on symmetrinen y-akselin suhteen.
%
%\textbf{Esimerkki 2.} \\
%Piirrä funktioiden $f(x)=x^2+1$, $g(x)=2x^2$ ja $h(x)=\frac{1}{3}x^2$ kuvaajat. \\ \\
%
%\begin{kuvaajapohja}{1}{-2}{2}{-1}{3}
%  \kuvaaja{x**2+1}{$f(x)=x^2+1$}{blue}
%\end{kuvaajapohja}
%
%
%\begin{kuvaajapohja}{1}{-2}{2}{-1}{3}
%  \kuvaaja{2*x**2}{$g(x)=2x^2$}{blue}
%\end{kuvaajapohja}
%
%\begin{kuvaajapohja}{1}{-2}{2}{-1}{3}
%  \kuvaaja{(1 / 3.0)*(x**2)}{$h(x)=\frac{1}{3}x^2$}{blue}
%\end{kuvaajapohja}
%
%Mitä tapahtuu funktion kuvaajan muodolle, kun termin $ax^2$ kerroin $a$ muuttuu? \\ \\
%
%\textbf{Esimerkki 3.} \\
%Piirrä funktioiden $f(x)=-x^2+x$, $g(x)=-x^2+2x+1$ ja $h(x)=-x^2+\frac{1}{2}x-1$ kuvaajat. \\
%\missingfigure \\
%Mitä tapahtuu funktion kuvaajan muodolle, kun termin $bx$ kerroin $b$ muuttuu? \\ \\
%
%\textbf{Esimerkki 4.} \\
%Piirrä funktioiden $f(x)=x^2$, $g(x)=x^2-2$ ja $h(x)=x^2+\frac{3}{2}$ kuvaajat. \\ \\
%Mitä tapahtuu funktion kuvaajan muodolle, kun vakiotermi $c$ muuttuu? \\ \\

Koottuna:

\laatikko{Toisen asteen polynomifunktion $f(x)=ax^2+bx+c$ kuvaaja on
\begin{itemize}
\item ylöspäin aukeava paraabeli, kun $a>0$
\item alaspäin aukeava paraabeli, kun $a<0$
\item sitä kapeampi, mitä suurempi $|a|$ on.
\end{itemize}
}

% FIXME: selitys tulisi hoitaa ilman itseisarvoa

%\textbf{Esimerkki 5.} \\
%Ratkaise funktion $f(x)=4x^2-13x+8$ nollakohdat.
%\begin{align*}
%f(x)&=0 \\
%4x^2-13x+8&=0 \\
%x&=\frac{-(-13) \pm \sqrt[]{(-13)^2-4 \cdot 4 \cdot 8}}{2 \cdot 4} \\
%x&=\frac{13 \pm \sqrt[]{169-128}}{8} \\
%x&=\frac{13 \pm \sqrt[]{41}}{8} \\
%x&=\frac{13 \pm \sqrt[]{41}}{8}
%\end{align*}




%Kuvaajaan transformaatioihin esimerkkiä myös muuttujan x muuttamisesta. Piirrä f(x+1):n kuvaaja jne.
%funktion nollakohtien ja yhtälön juurien yhteys
%paraabelin muodon perustelu: P(x)=ax^2+bx+c=a(x-b/2a)^2+b^2/4a^2, siis pienin/suurin arvo kun x=-b/2a
%-> ylöspäin ja alaspäin aukeavat paraabelit

% Tämä on vähän ongelmallinen, polynomien jakokulma tarvittaisiin
% \input{appendices/poljako_todistus}



%\section{Sekalaisia tehtäviä}

% Tämä tiedosto ei ole käytössä tälle hetkellä.

%\begin{tehtava}
%  \begin{enumerate}[a)]
%    \item Ratkaise funktion $2x^2 - 5x - 3$ nollakohdat
%    \item Millä muuttujan arvoilla edellisen kohdan funktio $2x^2 - 5x - 3$ saa positiivisia arvoja?
%    \item Onko em. funktiolla globaali ääriarvo (minimi tai maksimi), ja jos on, missä kohtaa funktio saa tämän arvon? Mikä on funktion arvo silloin?
%  \end{enumerate}
%
%  \begin{vastaus}
%    \begin{enumerate}[a)]
%      \item $x = 1.2$ tai $x = -0.2$
%      \item Kun $x<-0,2$ tai $x>1,2$
%      \item Koska neliötermin kerroin a on positiivinen (2), funktiolla on globaali minimi (mutta ei ylärajaa). Symmetrian vuoksi minimi on nollakohtien puolivälissä kohdassa 0.5, jossa funktio saa siis pienimmän arvonsa -5.
%    \end{enumerate}
%  \end{vastaus}
%\end{tehtava}

% tuo yllä oleva on jäänyt turhaksi.

\Closesolutionfile{ans}

\newpage
\section{Vastaukset}
\begin{vastaussivu}
\input{appendices/vastaukset}
\end{vastaussivu}

